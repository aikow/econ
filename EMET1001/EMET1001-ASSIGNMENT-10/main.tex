%This is a LaTeX template for homework assignments
\documentclass{article}

\usepackage[utf8]{inputenc}
\usepackage{amsmath}
\usepackage{amssymb}
\usepackage{graphicx}
\usepackage{enumitem}

\begin{document}

%\title{EMET1001: ASSIGNMENT WEEK 2}
%\author{ZEMING WANG - U6114134}


%\maketitle
%\vspace{1in}

%TUTOR: LONG

\thispagestyle{empty}

\begin{center}
\huge
\vspace*{1.0in} EMET1001 
\\\vspace{0.5in} ASSIGNMENT 10
\normalsize
\\\vspace{0.5in} \textsc{By}
\\\vspace{0.1in} \textsc{Zeming Wang}
\\\vspace{0.1in} \textsc{u6114134}
\normalsize
\\\vspace{0.5in} \textsc{Tutorial: Thur. 2-3 pm}
\\\vspace{0.1in} \textsc{Tutor: Long}
\normalsize
\\\vspace{0.5in} \textsc{Due: Oct. 10, 2016}
\end{center}

\newpage
\setcounter{page}{1}

\subsection*{Chapter 13 Problems}
\vspace{0.1in}

\begin{enumerate}
    \item[1. ] The function $f$ defined for all $(x, y)$ by $$f(x, y) = -2x^2 + 2xy - y^2 + 18x - 14y + 4$$ has a maximum. 
    Find the corresponding values of $x$ and $y$. Use the sufficient conditions to prove that is has a maximum point. \\
    
    \textbf{Solution}: First order conditions are:
    \begin{align*}
        f_x &= -4x + 2y + 18 &= 0 \\
        f_y &= 2x - 2y - 14  &= 0 \\
    \end{align*}
    Solve the equations, we get:
    $$ x = 2,\ y = -5 $$
    We can prove this is a maximum point by checking the second order conditions:
    \begin{align*}
        f_{xx} &= -4 < 0 \\
        f_{yy} &= -2 < 0 \\
        f_{xy} &= 2 \\   
        f_{xx} & f_{yy} - (f_{xy})^2 = 4 > 0
    \end{align*}
    Therefore, it is the maximum point. \\
    
    \item[4. ] Find the stationary points of the following functions:
        \begin{enumerate}
            \item[(a)] $ f(x,y) = x^3 - x^2y + y^2 $ \\
            
            \textbf{Solution}:
            \begin{align*}
                f_x &= 3x^2 - 2xy &= 0 \\
                f_y &= -x^2 + 2y  &= 0 \\                
            \end{align*}
            Solve for $(x,y)$, we have the stationary points:
            $$ (0,0),\ \textrm{or}\ \Big( 3, \frac{9}{2}\Big) $$

            \item[(b)] $ f(x,y) = 4y^3 + 12x^2y - 24x^2 - 24y^2 $ \\
            
            \textbf{Solution}:
            \begin{align*}
                f_x &= 24xy - 48x          &= 0 \\
                f_y &= 12y^2 + 12x^2 - 48y &= 0 \\
            \end{align*}
            
            Rearrange the first equation:
            $$  24x(y-2) = 0 $$
            Therefore,
            $$ x = 0,\ \textrm{or}\ y = 2 $$
            
            If $x = 0$, substitute it into the second equation:
            $$ 12y^2 - 48y = 0 $$
            which yields:
            $$ y=0,\ \textrm{or}\ y=4 $$
            
            If $y=2$, substitute it into the second equation:
            $$ 48 + 12x^2 - 96 = 0 $$
            which yields:
            $$ x=2,\ \textrm{or}\ x=-2 $$
            
            Therefore, we have $4$ stationary points:
            $$(0,0),\ (0,4),\ (2,2),\ (-2,2)$$
        \end{enumerate}
        
    \item[5. ] Define $$ f(x,y,a) = ax^2 -2x + y^2 - 4ay $$ where $a$ is a parameter. For each fixed $a \neq 0$, find the point $(x^*(a), y^*(a))$ that makes the function $f$ stationary with respect to $(x,y)$. Find also the value function $f^*(a)=f(x^*(a), y^*(a), a)$, and verify the envelope theorem in this case. \\
    
    \textbf{Solution}: Find the FOCs first:
    \begin{align*}
        f_x &= 2ax -2  &= 0 \\
        f_y &= 2y - 4a &= 0 \\
    \end{align*}
    Solve the equations and get the stationary point:
    \begin{align*}
        x^* &= \frac{1}{a} \\
        y^* &= 2a \\
    \end{align*}
    The value function is:
    $$ f^*(a) = f(x^*, y^*, a) = -\frac{1}{a}-4a^2 $$
    To verify the envelop theorem, differentiate the value function:
    $$ f^*_a(a) = \frac{1}{a^2} - 8a $$
    Differentiate the objective function:
    $$ f_a(x,y,a) = x^2 - 4y $$
    Substitute $(x^*, y^*)$ in:
    $$ f_a(x^*, y^*,a) = \frac{1}{a^2} - 8a $$
    we can see that:
    $$ f^*_a(a) = f_a(x^*, y^*, a)$$
    That verifies the envelope theorem.
      
\end{enumerate}


\subsection*{Chapter 14 Problems}
\vspace{0.1in}

\begin{enumerate}
    \item[1. ] 
        \begin{enumerate}
            \item[(a)] Solve the following problem using the Lagrange multiplier method: $$ \max{ f(x,y)=3x+4y }\ \textrm{subject to}\ g(x,y)=x^2+y^2 = 225 $$
            
            \textbf{Solution}: The Lagrangian is:
            $$ \mathcal{L}(x,y) = 3x+4y-\lambda(x^2+y^2-225) $$
            FOCs are:
            \begin{align*}
                \mathcal{L}_x &= 3 - 2\lambda x = 0 \\
                \mathcal{L}_y &= 4 - 2\lambda y = 0 \\
                x^2 &+ y^2 = 225 \\
            \end{align*}
            Solve the equation we have:
            \begin{align*}
                \lambda &= \frac{1}{6} \\
                x &= 9 \\
                y &= 12 \\
            \end{align*}
            
            \item[(b)] Suppose $225$ is changed to $224$. What is the approximate change in the optimal value of $f$. \\
            
            \textrm{Solution}: The Lagrange multiplier $\lambda$ at the solution is just the rate of change of the value function. So if $225$ is changed to $224$, the change of value function $f^*$ is approximately $\frac{1}{6}$.
            
        \end{enumerate}
    
    \item[6. ] Consider the utility maximization problem with separable utility function: $$ \max{u(x,y)=v(x)+w(y)}\ \textrm{subject to}\ px + qy = m $$ where $v'(x)>0,w'(y)>0,v''(x)\leq 0,w''(y)\leq 0$.
        \begin{enumerate}
            \item[(a)] State the first-order conditions for utility maximization.\\
            
            \textbf{Solution}: First-order conditions are:
            \begin{align*}
                \mathcal{L}_x &= v'(x) - \lambda p = 0 \\
                \mathcal{L}_y &= w'(y) - \lambda q = 0 \\
                px &+ qy = m \\
            \end{align*}
            
            \item[(b)] Why are these conditions sufficient for optimality? \\
            
            \textbf{Solution}: We can show these conditions are sufficient by showing that the Lagrangian is concave. \\
            
            Given that $v'(x)>0,w'(y)>0,v''(x)\leq 0,w''(y)\leq 0$, we know both $v(x)$ and $w(y)$ are concave. That is, for $k \in [0,1]$, we have:
            $$ v(kx_1 + (1-k)x_2) \geq kv(x_1) + (1-k)v(x_2) $$
            $$ w(ky_1 + (1-k)y_2) \geq kw(y_1) + (1-k)w(y_2) $$
            
            Therefore we have:
            \begin{align*}
                       \ & \mathcal{L}( kx_1 + (1-k)x_2, ky_1 + (1-k)y_2 ) 
                \\    =\ & v(kx_1 + (1-k)x_2) + w(ky_1 + (1-k)y_2) - \lambda(p(kx_1 + (1-k)x_2) + 
                \\       & q(ky_1 + (1-k)y_2) - m)
                \\ \geq\ & (kv(x_1) + (1-k)v(x_2)) + (kw(y_1) + (1-k)w(y_2)) - \lambda(p(kx_1 + (1-k)x_2) + 
                \\       & q(ky_1 + (1-k)y_2) - m)
                \\    =\ & k( v(x_1)+w(y_1)-\lambda(px_1+qy_1-m) ) + (1-k)( v(x_2)+w(y_2) - \lambda(px_2+qy_2-m) )
                \\    =\ & k\mathcal{L}(x_1,y_1) + (1-k)\mathcal{L}(x_2,y_2)
            \end{align*}
            which shows $\mathcal{L}$ is concave and thus the conditions are sufficient.
        \end{enumerate}

\end{enumerate}

\end{document}