%This is a LaTeX template for homework assignments
\documentclass{article}

\usepackage[utf8]{inputenc}
\usepackage{amsmath}
\usepackage{amssymb}
\usepackage{graphicx}
\usepackage{enumitem}

\begin{document}

%\title{EMET1001: ASSIGNMENT WEEK 2}
%\author{ZEMING WANG - U6114134}


%\maketitle
%\vspace{1in}

%TUTOR: LONG

\thispagestyle{empty}

\begin{center}
\huge
\vspace*{1.0in} EMET1001 
\\\vspace{0.5in} ASSIGNMENT 4
\normalsize
\\\vspace{0.5in} \textsc{By}
\\\vspace{0.1in} \textsc{Zeming Wang}
\\\vspace{0.1in} \textsc{u6114134}
\normalsize
\\\vspace{0.5in} \textsc{Tutorial: Thur. 2-3 pm}
\\\vspace{0.1in} \textsc{Tutor: Long}
\normalsize
\\\vspace{0.5in} \textsc{Due: Aug. 15, 2016}
\end{center}

\newpage
\setcounter{page}{1}

\begin{enumerate}
    
    \item[8.] 
    
        \begin{enumerate}
            \item[(a)] Find $y'$ when $y$ is given implicitly by the equation $$ x^2 - xy + 2y^2 = 7 $$
            
            Take derivative on both side of the equation with respect to $x$:
            $$ 2x - y - xy' + 2yy' = 0 $$
            
            Solve for $y'$, we get:
            $$ y' = \frac{2x - y}{x - 2y} $$
            
            \item[(b)] Find the points where the graph has horizontal tangent and the points where it has vertical tangent. \\
            
            The points where the graph has horizontal tangent must have $y'$ equaling $0$ at that points.\\
            
            We know, $y'=0$ when $ 2x = y $. Substitute $y$ using $2x$ into the original equation:
            $$ x^2 - x \cdot 2x + 2(2x)^2 = 7 $$
            
            Solve the equation, we get $x = \pm 1$.\\
            
            Therefore, there are two points: $(1, 2)$ and $(-1, -2)$, where the graph has horizontal tangent. \\
            
            To find the points where the graph has vertical tangent, we first find derivative $x'$ respect to $y$:
            $$ 2xx' - x'y - x + 4y = 0 $$
            
            Solve for $x'$:
            $$ x' = \frac{x - 4y}{2x - y}$$
            
            Let $x'=0$, we get $x = 4y$. Substitute this to the original equation:
            $$ (4y)^2 - 4y^2 + 2y^2 = 7 $$
            
            Solve $y$ we get $y = \pm \dfrac{\sqrt{2}}{2}$, and thus, $x = \pm 2\sqrt{2}$ \\
            
            Therefore, the two points where the graph has vertical tangent line are: $(2\sqrt{2}, \frac{\sqrt{2}}{2})$ and $(-2\sqrt{2}, -\frac{\sqrt{2}}{2})$.
            
        \end{enumerate}
        
    \item[12.] Find the quadratic approximations to the following functions about $x = 0$.
    
        \begin{enumerate}
            \item[(a)] $ f(x) = \ln{(2x + 4)} $ \\
            
            The quadratic approximation of $f(x)$ at $x=0$ is given by:
            $$ f(x) \approx f(0) + f'(0)x + \frac{f''(0)}{2}x^2 $$
            
            So we have to figure out $f(0)$, $f'(0)$ and $f''(0)$ respectively.
            $$ f(x) = \ln{(2x + 4)} \implies f(0) = \ln{4} $$
            $$ f'(x) = \frac{1}{x+2} \implies f'(0) = \frac{1}{2} $$
            $$ f''(0) = -\frac{1}{(x+2)^2} \implies f''(0) = -\frac{1}{4} $$
           
            Therefore, the approximation is:
            $$ f(x) \approx \ln{4} + \frac{1}{2}x - \frac{1}{8}x^2 $$
            
            \item[(b)] $ g(x) = (1 + x)^{-\frac{1}{2}} $ \\
            
            Follow the similar steps as above:
            $$ g(x) = (1+x)^{-\frac{1}{2}} \implies g(0) = 1 $$
            $$ g'(x) = -\frac{1}{2}(1+x)^{-\frac{3}{2}} \implies g'(0) = -\frac{1}{2} $$
            $$ g''(x) = \frac{3}{4}(1+x)^{-\frac{5}{2}} \implies g''(0) = \frac{3}{4} $$
            
            Therefore, the approximation of $g(x)$ is:
            $$ g(x) \approx 1 - \frac{1}{2}x + \frac{3}{8}x^2 $$
            
            \item[(c)] $ h(x) = xe^{2x} $
            
            $$ h(x) = xe^{2x} \implies h(0) = 0 $$
            $$ h'(x) = e^{2x} + 2xe^{2x} \implies h'(0) = 1 $$
            $$ h''(x) = 2e^{2x} + 2h'(x) \implies h''(0) = 4 $$
            
            Therefore, the approximation of $h(x)$ is:
            $$ h(x) \approx x + 2x^2 $$
            
        \end{enumerate}
    
    \item[20.] Find the elasticities of the functions given by the following formulas.
    
        \begin{enumerate}
            \item[(a)] $ 50x^5 $ \\
            
            We know the elasticity of power function $y=Ax^a$ is just $a$. 
            
            Therefore the elasticity of this formula is simply:
            $$ El = 5 $$
            
            \item[(b)] $ \sqrt[3]{x} $
            
            $$ El = \frac{1}{3} $$
            
            \item[(c)] $ x^3 + x^5 $
            \begin{equation*}
            \begin{aligned}
                El &= \frac{x}{y}\cdot\frac{dy}{dx} \\
                   &= \frac{x}{x^3+x^5}\cdot(3x^2+5x^4) \\
                   &= \frac{5(x^3+x^5)-2x^3}{x^3+x^5} \\
                   &= 5 - \frac{2}{x^2+1}
            \end{aligned}
            \end{equation*}
            
            \item[(d)] $ \dfrac{x-1}{x+1} $
            \begin{equation*}
            \begin{aligned}
                El &= \frac{x}{y}\cdot\frac{dy}{dx} \\
                   &= \frac{x}{\frac{x-1}{x+1}}\cdot\frac{(x+1)-(x-1)}{(x+1)^2} \\
                   &= x\cdot\frac{x+1}{x-1}\cdot\frac{2}{(x+1)^2} \\
                   &= \frac{2x}{x^2-1}
            \end{aligned}
            \end{equation*}
        \end{enumerate}
    
    \item[23.] Evaluate the limits:
        \begin{enumerate}
            \item[(a)] $ \lim_{x \rightarrow 3^{-}} (x^2-3x+2) $ \\
            
            Function $f(x) = x^2-3x+2$ is continuous at $x = 3$, so the limit is just the the value $f(3)$:
            $$ \lim_{x \rightarrow 3^{-}} (x^2-3x+2) = 3^2 -3 \cdot 3 + 2 = 2 $$
            
            \item[(b)] $ \lim_{x \rightarrow -2^{+}} \dfrac{x^2-3x+14}{x+2} $\\
            
            As $x \rightarrow -2^{+}$, the numerator approaches constant value $24$. While the denominator approaches $0^{+}$. Therefore the entire fraction approaches $+\infty$.
            $$ \lim_{x \rightarrow -2^{+}} \dfrac{x^2-3x+14}{x+2} = \infty $$
            
            \item[(c)] $ \lim_{x \rightarrow -1} \dfrac{3-\sqrt{x+17}}{x+1} $\\
            
            As $x \rightarrow -1$, the numerator approaches $3-\sqrt{16}=-1$, which is a fixed number. But it makes a difference to the denominator whether $x$ approaches $-1$ from left or right.\\
            
            As $x \rightarrow -1^{+}$, $x+1 \rightarrow 0^{+}$, which drives the whole fraction to $-\infty$. 
            As $x \rightarrow -1^{-}$, $x+1 \rightarrow 0^{-}$, which drives the whole fraction to $+\infty$. \\
            
            Since the left limit does not equal the right limit, the limit does not exist.
            
            \item[(f)] $ \lim_{x \rightarrow 4} \dfrac{x-4}{2x^2-32} $
            
            \begin{equation*}
            \begin{aligned}
                \lim_{x \rightarrow 4} \dfrac{x-4}{2x^2-32} 
                &= \lim_{x \rightarrow 4} \dfrac{x-4}{2(x+4)(x-4)} \\
                &= \lim_{x \rightarrow 4} \dfrac{1}{2(x+4)} \\
                &= \frac{1}{16}
            \end{aligned}
            \end{equation*}
            
            \item[(i)] $ \lim_{x \rightarrow \infty} \dfrac{(\ln{x})^2}{3x^2} $
            
            \begin{equation*}
            \begin{aligned}
                \lim_{x \rightarrow \infty} \dfrac{(\ln{x})^2}{3x^2}
                &= \lim_{x \rightarrow \infty} \bigg( \dfrac{\ln{x}}{\sqrt{3}x} \bigg)^2 \\
                &= \bigg( \lim_{x \rightarrow \infty} \dfrac{\ln{x}}{\sqrt{3}x} \bigg)^2 \\
                &= \bigg( \lim_{x \rightarrow \infty} \dfrac{1/x}{\sqrt{3}} \bigg)^2,\ \textrm{by L'Hospital rule} \\
                &= (0)^2 = 0
            \end{aligned}
            \end{equation*}
            
        \end{enumerate}
    
    \item[25.] Evaluate $$ \lim_{x \rightarrow 0} \frac{a^x-b^x}{e^{ax}-e^{bx}} $$ 
    $a \neq b$, $a$ and $b$ positive. \\
    
    The Taylor series equivalence of exponential functions is:
    $$ e^x = \sum_{n=0}^{\infty} \dfrac{1}{n!}x^n $$
    
    Since $a^x = e^{\ln{a^x}} = e^{x\ln{a}}$, we have:
    $$ a^x \approx 1 + \ln{a}\cdot x + O(|x|^2) $$
    $$ b^x \approx 1 + \ln{b}\cdot x + O(|x|^2) $$ 
    
    Meanwhile,
    $$ e^{ax} \approx 1 + ax + O(|x|^2) $$
    $$ e^{bx} \approx 1 + bx + O(|x|^2) $$
    
    Substitute these polynomial approximations into the original formula:
    \begin{equation*}
    \begin{aligned}
        \lim_{x \rightarrow 0} \frac{a^x-b^x}{e^{ax}-e^{bx}}
        &= \lim_{x \rightarrow 0} \frac{(1+\ln{a}\cdot x) - (1+\ln{b}\cdot x)}{(1+ax) - (1+bx)} \\
        &= \lim_{x \rightarrow 0} \frac{(\ln{a} - \ln{b})x}{(a - b)x} \\
        &= \lim_{x \rightarrow 0} \frac{\ln{a} - \ln{b}}{a - b} \\
        &= \frac{\ln{a} - \ln{b}}{a - b}
    \end{aligned}
    \end{equation*}
    which is the final answer.
    
\end{enumerate}

\end{document}