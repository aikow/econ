%This is a LaTeX template for homework assignments
\documentclass{article}

\usepackage[utf8]{inputenc}
\usepackage{amsmath}
\usepackage{amssymb}
\usepackage{graphicx}
\usepackage{enumitem}

\begin{document}

%\title{EMET1001: ASSIGNMENT WEEK 2}
%\author{ZEMING WANG - U6114134}


%\maketitle
%\vspace{1in}

%TUTOR: LONG

\thispagestyle{empty}

\begin{center}
\huge
\vspace*{1.0in} EMET1001 
\\\vspace{0.5in} ASSIGNMENT 3
\normalsize
\\\vspace{0.5in} \textsc{By}
\\\vspace{0.1in} \textsc{Zeming Wang}
\\\vspace{0.1in} \textsc{u6114134}
\normalsize
\\\vspace{0.5in} \textsc{Tutorial: Thur. 2-3 pm}
\\\vspace{0.1in} \textsc{Tutor: Long}
\normalsize
\\\vspace{0.5in} \textsc{Due: Aug. 8, 2016}
\end{center}


\newpage
\setcounter{page}{1}


\begin{enumerate}

    \item[5.] Find the equation for the tangent to the graph at the specified point.
    
        \begin{enumerate}
            
            \item[(a)] $f(x) = -3x^2$ at $x = 1$ \\
            
            The equation of the tangent line to the function $f(x)$ at point $c$ is: 
            $$ y - f(c) = f'(c)(x - c)$$
                
            To find the equation of the tangent, we need to calculate $f(c)$, $f'(c)$.
            
            It is easy to calculate:
            $$ f(1) = -3 \cdot 1^2 = -3 $$
            
            To calculate $f'(1)$, we first find the derivative of $f(x)$:
            $$ f'(x) = -6x $$
            
            Substitute $x = 1$ in, we get:
            $$ f'(1) = -6 $$
            
            Therefore, the equation of the tangent line is given by:
            $$ y - f(1) = f'(1)(x - 1) $$
            $$ y + 3 = -6(x - 1) $$
            
            which is:
            $$ y = -6x + 3$$
            
            \item[(b)] $f(x) = \sqrt{x} - x^2$ at $x = 4$
            
            $$ f'(x) = \frac{1}{2\sqrt{x}} - 2x $$
            $$ f(4) = \sqrt{4} - 4^2 = -14 $$
            $$ f'(4) = \frac{1}{2\sqrt{4}} - 2 \cdot 4 = -\frac{31}{4} $$
            
            The equation of the tangent line is:
            $$ y - (-14) = -\frac{31}{4}(x - 4) $$
            
            Simplify the equation:
            $$ y = -\frac{31}{4}x + 17 $$
            
            
            \item[(c)] $f(x) = \dfrac{x^2 - x^3}{x + 3}$ at $x = 1$ \\
            
            Following the rule of derivative of a quotient:
            $$ f'(x) = \frac{(2x - 3x^2)(x + 3) - (x^2 - x^3)}{(x + 3)^2} = \frac{-2x^3 - 8x^2 + 6x}{(x + 3)^2} $$
            
            So we have:
            $$ f'(1) = \frac{-2-8+6}{4^2} = -\frac{1}{4} $$
            
            It is easy to know:
            $$ f(1) = 0 $$
            
            Therefore the equation of the tangent line is:
            $$ y = -\frac{1}{4}(x - 1) = -\frac{1}{4}x + \frac{1}{4} $$
            
        \end{enumerate}
    
    
    \item[7.] Differentiate the following functions. 
    
        \begin{enumerate}
            
            \item[(a)] $ f(x) = x(x^2 + 1) $
            
            $$ f(x) =  x(x^2 + 1) = x^3 + x $$
            
            $$ \frac{d}{dx} f(x) = 3x^2 + 1 $$
            
            \item[(b)] $ g(w) = w^{-5} $
            
            $$ \frac{d}{dw} g(w) = -5w^{-6} $$
            
            \item[(c)] $ h(y) = y(y-1)(y+1) $
            
            $$ h(y) = y(y-1)(y+1) = y(y^2 - 1) = y^3 - y $$
            
            $$ \frac{d}{dy} h(y) = 3y^2 - 1 $$
            
            \item[(d)] $ G(t) = \dfrac{2t + 1}{t^2 + 3} $
            
            $$ \frac{d}{dt} G(t) = \frac{ 2(t^2+3) - 2t(2t+1) }{ (t^2 + 3)^2 } = \frac{ -2t^2 - 2t + 6 }{ t^4 + 6t^2 + 9 }$$
            
            \item[(e)] $ \psi(\xi) = \dfrac{2\xi}{\xi^2 + 2} $
            
            $$ \frac{d}{d\xi} \psi(\xi) = \frac{ 2(\xi^2+2) - 2\xi \cdot 2\xi }{ (\xi^2+2)^2 } = \frac{ -2\xi^2+4 }{ \xi^4+4\xi^2+4 } $$
            
            \item[(f)] $ F(s) = \dfrac{s}{s^2 + s - 2} $
            
            $$ \frac{d}{ds} F(s) = \frac{ (s^2+s-2) - s(2s+1) }{ (s^2+s-2)^2 } = -\frac{ s^2+2 }{ (s^2+s-2)^2 } $$
            
        \end{enumerate}
        
        
    \item[9.] Use the chain rule to find $\frac{dy}{dx}$ for the following.
    
        \begin{enumerate}
            
            \item[(a)] $ y = 10u^2 $ and $ u = 5 - x^2 $
            
            $$ \frac{dy}{dx} = \frac{dy}{du} \cdot \frac{du}{dx} = 20u \cdot (-2x) = 20( 5 - x^2 )(-2x) = 40x^3-200x $$
            
            \item[(b)] $ y = \sqrt{u} $ and $ u = \dfrac{1}{x} - 1 $
            
            $$ \frac{dy}{dx} = \frac{dy}{du} \cdot \frac{du}{dx} = \frac{1}{2\sqrt{u}} \cdot \Big( -\frac{1}{x^2} \Big) = \frac{1}{2\sqrt{\frac{1}{x} - 1}} \cdot \Big( -\frac{1}{x^2} \Big) = -\frac{1}{2\sqrt{ x^3 - x^4 }} $$
            
        \end{enumerate}
    
    
    \item[14.] Find the first-order derivatives of:
    
        \begin{enumerate}
            
            \item[(a)] $ y = -7e^x $
            
            $$ \frac{dy}{dx} =  -7e^x $$
            
            \item[(b)] $ y = e^{-3x^2} $ \\
            
            Let $ u = -3x^2 $, $ y = e^u $. Then we have:
            
            $$ \frac{dy}{dx} = \frac{dy}{du} \cdot \frac{du}{dx} = e^u \cdot (-6x) = -6x e^{-3x^2} $$
            
            \item[(c)] $ y = \dfrac{x^2}{e^x} $
            
            $$ \frac{dy}{dx} = \frac{ 2x \cdot e^x - x^2 \cdot e^x }{ (e^x)^2 } = \frac{ (2x-x^2)e^x }{ (e^x)^2 } = \frac{ 2x-x^2 }{ e^x } $$
            
            \item[(d)] $ y = e^x \ln{ (x^2 + 2) } $ \\
            
            Let $ u = \ln{ (x^2 + 2) } $, then $ y = e^x u $. 
            
            $$ dy = d(e^x) \cdot u + e^x \cdot du = e^x dx \cdot u + e^x \cdot du $$
            
            Let $ v = x^2 + 2 $, so $ u = \ln{v} $.
            
            $$ du = d(\ln{v}) = \frac{1}{v} dv = \frac{1}{v} \cdot 2xdx = \frac{2x}{x^2+2}dx $$
            
            Substitute $du$ back to the first equation:
            
            $$ dy = e^x dx \cdot \ln{ (x^2+2) } + e^x \cdot \frac{2x}{x^2+2}dx $$
            
            Therefore,
            
            $$ \frac{dy}{dx} = e^x \Big[ \ln{ (x^2+2) } + \frac{2x}{x^2 + 2} \Big] $$
            
            \item[(e)] $ y = e^{ 5x^3 } $
            
            $$ \frac{dy}{dx} = e^{ 5x^3 } \cdot (15x^2) = 15x^2 e^{ 5x^3 } $$
            
            \item[(f)] $ y = 2 - x^4e^{-x} $
            
            $$ \frac{dy}{dx} = -4x^3 e^{-x} - x^4 e^{-x} (-1) = x^3e^{-x}(x - 4) $$
            
            \item[(g)] $ y = (e^x + x^2)^{10} $ \\
            
            Let $ u = e^x + x^2 $, then $ y = u^{10} $.
            
            $$ dy = d(u^{10}) = 10u^9du $$
            
            While
            
            $$ du = d(e^x) + d(x^2) = e^x dx + 2x dx = (e^x + 2x) dx $$
            
            Substitute $u$ and $du$ into the first equation:
            
            $$ dy = 10(e^x + x^2)^9 \cdot (e^x + 2x) dx $$
            
            Therefore,
            
            $$ \frac{dy}{dx} = 10(e^x + x^2)^9 (e^x + 2x) $$
            
            \item[(h)] $ y = \ln{ \sqrt{x} + 1 } $
                
            $$ \frac{dy}{dx} = \frac{ \frac{1}{ 2\sqrt{x} }}{ \sqrt{x} } = \frac{1}{2x} $$
            
        \end{enumerate}
    
    
    
    \item[15.] Find the intervals where the following functions are increasing.
    
        \begin{enumerate}
            
            \item[(a)] $ y = (\ln{x})^2 - 4 $ \\
            
            If we let $u = \ln{x}$, then $y = u^2 - 4$, which is a quadratic function.
            
            We know the increasing interval for the quadratic function is $[0, \infty)$.
            
            So we just need to figure out the interval for $x$ which makes $u  \geq 0$. \\
            
            
            As we know, $u = \ln{x}$ is strictly increasing on $(0, \infty)$, and $u = 0$ when $x = 1$.
            So when $x \geq 1$, we have $ u \geq 0$. \\
            
            Therefore, $[1, \infty)$ is the interval on which $y$ is increasing. \\
            
            \item[(b)] $ y = \ln{ (e^x + e^{-x}) }$ \\
            
            This function is more complicated and we cannot figure out the increasing interval by observation. \\
            
            But we know the function is increasing \textit{iff} the derivative $y' \geq 0$. \\
            
            So let's first calculate the derivative of $y$:
            
            $$ y' = \frac{ e^x - e^{-x} }{ e^x + e^{-x} } $$
            
            Since $e^x > 0$ for all $x$, $ e^x + e^{-x} > 0 $ always hold. So we have:
            
            $$ y' \geq 0 \iff e^x - e^{-x} \geq 0 $$
            
            Multiply both side by $e^x$:
            
            $$ e^{2x} - 1 \geq 0 \iff e^{2x} \geq 1 $$
            
            We know that $e^x$ is strictly increasing, and $e^x = 1$ when $x = 0$. Therefore,
            
            $$ e^{2x} \geq 1 \iff 2x \geq 0 \iff x \geq 0 $$
            
            Therefore, on the interval $[0, \infty)$, $y' \geq 0$, and thus $y$ is increasing. \\
            
            \item[(c)] $ y = x - \dfrac{3}{2}\ln{(x^2+2)} $ \\
            
            Following the similar steps in (b), first take the derivative:
            
            $$ y' = 1 - \frac{3}{2} \cdot \frac{2x}{x^2 + 2} = 1 - \frac{3x}{x^2 + 2} $$
            
            $y$ is increasing means $y' \geq 0$:
            
            \begin{equation*}
            \begin{aligned}
                1 - \frac{3x}{x^2 + 2} \geq 0 &\iff
                x^2 + 2 - 3x \geq 0 \\ &\iff 
                \Big( x + \frac{3}{2} \Big)^2 - \frac{1}{4} \geq 0 \\ &\iff 
                \Big( x + \frac{3}{2} \Big)^2 \geq \frac{1}{4} \\ &\iff
                x + \frac{3}{2} \geq \frac{1}{2}, \ \textrm{or}\  x + \frac{3}{2} \leq -\frac{1}{2} \\ &\iff
                x \geq -1,\ \textrm{or}\ x \leq -2
            \end{aligned}
            \end{equation*}
            
            Therefore, $(-\infty, -2] \cup [-1, \infty)$ is the increasing interval.
           
            %\begin{flushright}$(\#)$\end{flushright}
            
        \end{enumerate}
    
    \end{enumerate}

    
\end{document}