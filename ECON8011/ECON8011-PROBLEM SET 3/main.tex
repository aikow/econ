\documentclass{article}
\usepackage{amssymb}
\usepackage{amsthm}
\usepackage{amsmath}
\usepackage{graphicx}
\usepackage{enumerate}
\usepackage{array,multirow}
\usepackage[usenames, dvipsnames]{xcolor}
\usepackage{bm}

\title{PROBLEM SET 3}
% \author{
%     \textsc{Zeming Wang}
%     \thanks{u6114134@anu.edu.au}
% }
\date{}

\setlength\parindent{0pt}

\begin{document}

\maketitle

\section*{Question 1}
Consider the single-output technology whose production function is given by
\[ f(z)=(z_1^{\rho} + z_2^{\rho})^{\frac{1}{\rho}} \]
for $\rho \leq 1$

\begin{enumerate}
    \item Derive the profit function $\pi(p)$ and supply function $y(p)$. \\

    The profit maximization problem is
    \[ \underset{z\geq 0}{\max}\ pf(z) - w\cdot z\]

    Notice that $f(z)$ has constant return to scale:
    \[ f(\lambda z)
       =\left((\lambda z_1)^{\rho} + (\lambda z_2)^{\rho}\right)^{\frac{1}{\rho}}
       =\lambda (z_1^{\rho} + z_2^{\rho})^{\frac{1}{\rho}}
       =\lambda f(z)
    \]
    Therefore, the firms will either have $y(p)=\pi(p)=0$,
    or $y(p)=\pi(p)=+\infty$, depending on the value of $p$ and $w$.
    Because if there exists some $z$ that makes $\pi = pf(z)-w\cdot z > 0$, 
    the firm could arbitrarily scale up production and the profit explodes to 
    infinity. 
    $pf(\lambda z)-w\cdot \lambda z = \lambda (pf(z)-w\cdot z) \to +\infty$ 
    as $\lambda \to +\infty$. 
    If the prevailing $p$ and $w$ are such that $\pi = pf(z)-w\cdot z \leq 0$ 
    for all $z$, the firm would rather produce nothing and the profit is zero. \\
    
    % Suppose $p > 0$ and $w >> 0$.\\
    %
    % Firstly, if $\rho=1$, the problem boils down to
    % \[ \underset{z\geq 0}{\max}\ p (z_1+z_2) - w_1z_1 - w_2z_2\]
    %
    % If $p \leq \min\{w_1,w_2\}$, the optimal choice for the firm is to product
    % nothing. In this case, $y(p)=0$, $\pi(p)=0$.If $p > w_i$, $i \in \{1,2\}$,
    % the firm will product $z_i$ as much as possible and $\pi(p)=+\infty$. \\
    %
    % Now consider the case $p<0$.
    % The first-order condition is
    % \[ \nabla f(z) = \frac{w}{p} \]
    % Or equivalently,
    % \begin{equation}
    %     \label{eqn:z1}
    %     (z_1^{\rho} + z_2^{\rho})^{\frac{1}{\rho}}\ z_1 =
    %       \left(\frac{w_1}{p}\right)^{\frac{1}{\rho-1}}
    % \end{equation}
    % \begin{equation}
    %     \label{eqn:z2}
    %     (z_1^{\rho} + z_2^{\rho})^{\frac{1}{\rho}}\ z_2 =
    %       \left(\frac{w_2}{p}\right)^{\frac{1}{\rho-1}}
    % \end{equation}
    % Notice that, since $f(z)$ is concave, the first-order condition is also
    % sufficient. \\
    %
    % (\ref{eqn:z1}) / (\ref{eqn:z2}) gives:
    % \begin{equation}
    %     \label{eqn:div}
    %     \frac{z_1}{z_2} = \left(\frac{w_1}{w_2}\right)^{\frac{1}{\rho-1}}
    % \end{equation}
    % Rearrange (\ref{eqn:z1}) and replace (\ref{eqn:div}) into (\ref{eqn:z1}),
    % \[ \left[\left(\frac{z_1}{z_2}\right)^{\rho}+1\right]^{\frac{1}{\rho}}\
    %     z_1 z_2 = \left(\frac{w_1}{p}\right)^{\frac{1}{\rho-1}}
    % \]
    % \[ \left[
    %       \left(\frac{w_1}{w_2}\right)^{\frac{\rho}{\rho-1}}+1
    %     \right]^{\frac{1}{\rho}}\ z_1 z_2 =
    %     \left(\frac{w_1}{p}\right)^{\frac{1}{\rho-1}}
    % \]
    % Thus we have,
    % \begin{equation}
    %     \label{eqn:mul}
    %     z_1z_2 = p^{-\frac{1}{\rho-1}}
    %       \left(w_1^{\frac{\rho}{\rho-1}} + w_2^{\frac{\rho}{\rho-1}}\right)^
    %          {-\frac{1}{\rho}}(w_1w_2)^{\frac{1}{\rho-1}}
    % \end{equation}
    %
    % (\ref{eqn:div}) $\times$ (\ref{eqn:mul}) and solve for $z_1$:
    % \[ z_1 = \sqrt{p^{-\frac{1}{\rho-1}}
    %       \left(w_1^{\frac{\rho}{\rho-1}} + w_2^{\frac{\rho}{\rho-1}}\right)^
    %          {-\frac{1}{\rho}}}\ w_1^{\frac{1}{\rho-1}}
    % \]
    %
    % (\ref{eqn:div}) / (\ref{eqn:mul}) and solve for $z_2$:
    % \[ z_2 = \sqrt{p^{-\frac{1}{\rho-1}}
    %       \left(w_1^{\frac{\rho}{\rho-1}} + w_2^{\frac{\rho}{\rho-1}}\right)^
    %           {-\frac{1}{\rho}}}\ w_2^{\frac{1}{\rho-1}}
    % \]
    %
    % Substitute $z_1,z_2$ into $f(z)$ gives the \emph{supply function}:
    % \[ y(p) = f(z(w,p)) = \sqrt{p^{-\frac{1}{\rho-1}}
    %       \left(w_1^{\frac{\rho}{\rho-1}} + w_2^{\frac{\rho}{\rho-1}}\right)^
    %           {\frac{1}{\rho}}}
    % \]
    %
    % The \emph{profit function} is therefore,
    % \begin{align*}
    %     \pi(p) &= p y(p) - w_1z_1 - w_2z_2 \\
    %            &= p \sqrt {
    %                     p^{-\frac{1}{\rho-1}}
    %                     \left(
    %                         w_1^{\frac{\rho}{\rho-1}} +
    %                         w_2^{\frac{\rho}{\rho-1}}
    %                     \right)^{\frac{1}{\rho}}
    %                 } - \sqrt {
    %                     p^{-\frac{1}{\rho-1}}
    %                     \left(
    %                         w_1^{\frac{\rho}{\rho-1}} +
    %                         w_2^{\frac{\rho}{\rho-1}}
    %                     \right)^{-\frac{1}{\rho}}
    %                 }
    %                 \left(
    %                     w_1^{\frac{\rho}{\rho-1}} +
    %                     w_2^{\frac{\rho}{\rho-1}}
    %                 \right) \\
    %            &= \sqrt {
    %                    p^{-\frac{1}{\rho-1}}
    %                    \left( w_1^{\frac{\rho}{\rho-1}} +
    %                           w_2^{\frac{\rho}{\rho-1}}
    %                    \right)^{-\frac{1}{\rho}}
    %               }
    %               \left[ p \left( w_1^{\frac{\rho}{\rho-1}} +
    %                               w_2^{\frac{\rho}{\rho-1}}
    %                       \right)^{\frac{1}{\rho}} -
    %                       \left( w_1^{\frac{\rho}{\rho-1}} +
    %                              w_2^{\frac{\rho}{\rho-1}}
    %                       \right)
    %               \right]
    % \end{align*}
    %
\item Derive the cost function $c(w,q)$ and conditional factor demand
      correspondence $z(w,q)$.\\

The cost minimization problem is
\[ \underset{z\geq 0}{\min}\ w \cdot z\quad \textrm{s.t.}\ f(z) \geq q \]

First consider the case $\rho=1$. The problem boils down to 
\[ \underset{z\geq 0}{\min}\ w_1z_1+w_2z_2 \quad \textrm{s.t.}\ z_1+z_2 \geq q \]

In this case, the firm will produce entirely with whichever input is cheaper. 
So it is easy to conclude
\[ 
    z(w,q) = 
    \begin{cases}
        (q,0)\quad \textrm{if } w_1 < w_2 \\
        (0,q)\quad \textrm{if } w_1 > w_2 \\
        \{ (z_1,z_2): z_1+z_2=q \}\quad \textrm{if } w_1=w_2
    \end{cases}
\]
The cost function is therefore,
\[ 
    c(w,q) = 
    \begin{cases}
        w_1 q \quad \textrm{if } w_1 < w_2 \\
        w_2 q \quad \textrm{if } w_1 \geq w_2 
    \end{cases}
\]

For the case $\rho<1$. Set up a Lagrangian to solve the problem.
\[ \mathcal{L} = w \cdot z - \lambda (f(z)-q)\]

Suppose $w >> 0$. 
Since $w\cdot z$ and $f(z)$ are both strictly increasing functions, 
the inequality constraint $f(z)\geq q$ must hold with equality in optimal solutions. 
Assume at the moment the non-negative constraint $z\geq 0$ is not binding, and 
discuss this later. \\

The first-order conditions are:
\begin{align*}
    w - \lambda \nabla f(z) &= 0 \\
    f(z) - q &= 0
\end{align*}
Or equivalently, 
\begin{align}
   \lambda (z_1^{\rho}+z_2^{\rho})^{\frac{1}{\rho}-1}z_1^{\rho-1} &= w_1 \label{eqn:z1}\\
   \lambda (z_1^{\rho}+z_2^{\rho})^{\frac{1}{\rho}-1}z_2^{\rho-1} &= w_2 \label{eqn:z2}\\
   (z_1^{\rho} + z_2^{\rho})^{\frac{1}{\rho}} &= q \label{eqn:q}
\end{align}

We can derive the following relationship by (\ref{eqn:z1})/(\ref{eqn:z2}):
\begin{equation}
    \label{eqn:div2}
    \frac{z_1}{z_2} = \left(\frac{w_1}{w_2}\right)^{\frac{1}{\rho-1}}
\end{equation}

Rearrange (\ref{eqn:q}) and substitute (\ref{eqn:div2}) in:
\[ (z_1^{\rho} + z_2^{\rho})^{\frac{1}{\rho}} = q \]
\[ \left[ \left( \frac{z_1}{z_2} \right)^{\rho} + 1
   \right]^{\frac{1}{\rho}}\ z_2 = q
\]
\[ \left[ \left(\frac{w_1}{w_2}\right)^{\frac{\rho}{\rho-1}} + 1
   \right]^{\frac{1}{\rho}}\ z_2 = q
\]

Solve for $z_2$ gives
\begin{align*}
  z_2 &= q \left[\left(\frac{w_1}{w_2}\right)^
           {\frac{\rho}{\rho-1}}+1\right]^{-\frac{1}{\rho}} \\
      &= q \left(w_1^{\frac{\rho}{\rho-1}} + w_2^{\frac{\rho}{\rho-1}}\right)^
           {-\frac{1}{\rho}}\ w_2^{\frac{1}{\rho-1}}
\end{align*}

Similarly, solve for $z_1$ gives
\[ z_1 = q \left(w_1^{\frac{\rho}{\rho-1}} + w_2^{\frac{\rho}{\rho-1}}\right)^
             {-\frac{1}{\rho}}\ w_1^{\frac{1}{\rho-1}}
\]

For $q>0$ and $w>>0$, we always have $z>>0$, so the non-negative constraint 
never binds, i.e. there is no corner solution for $\rho<1$. \\

Thus, the \emph{conditional factor demand correspondence} is
\[ z = q \left(w_1^{\frac{\rho}{\rho-1}} + w_2^{\frac{\rho}{\rho-1}}\right)^
              {-\frac{1}{\rho}}\
        \left(w_1^{\frac{1}{\rho-1}}, w_2^{\frac{1}{\rho-1}} \right)
\]

Therefore, the \emph{cost function} is
\[ c(w,q) = w_1z_1 + w_2z_2
          = q\left( w_1^{\frac{\rho}{\rho-1}} +
                   w_2^{\frac{\rho}{\rho-1}}
            \right)^{1-\frac{1}{\rho}} 
\]
\end{enumerate}

\section*{Question 2}
Consider the utility function
\[ u(x_1,x_2) = 2x_1^{\frac{1}{2}} + 4x_2^{\frac{1}{2}} \]
\begin{enumerate}
  \item Write the consumer's expenditure minimization problem.\\

  The consumer's expenditure minimization problem is
  \begin{align*}
    \underset{x \geq 0}{\min}\ & p_1x_1 + p_2x_2\quad \\
    &\textrm{s.t.}\ 2x_1^{\frac{1}{2}} + 4x_2^{\frac{1}{2}} \geq v
  \end{align*}

  \item Derive the expenditure problem for this problem.\\

  Notice that the utility constraint must be binding. 
  And also assume at the moment the non-negative constraint does not bind. 
  Define the Lagrange function:
  \[  \mathcal{L} = p_1x_1 + p_2x_2 - \lambda
       (2x_1^{\frac{1}{2}} + 4x_2^{\frac{1}{2}} - v)
  \]

  The first-order conditions are:
  \begin{align}
    p_1 - \lambda x_1^{-\frac{1}{2}} &= 0 \label{eqn:c1} \\
    p_2 - 2\lambda x_2^{-\frac{1}{2}} &= 0 \label{eqn:c2} \\
    2x_1^{\frac{1}{2}} + 4x_2^{\frac{1}{2}} - v &= 0 \label{eqn:c3}
  \end{align}

  From (\ref{eqn:c1}) and (\ref{eqn:c2}), we can derive
  \[ x_1^{\frac{1}{2}} =  \frac{\lambda}{p_1}, \quad
     x_2^{\frac{1}{2}} =  \frac{2\lambda}{p_2} \]

  Substitute them into (\ref{eqn:c3}) and solve for lambda,
  \[ \frac{2\lambda}{p_1} + \frac{8\lambda}{p_2} = v \]
  \[  \lambda = \frac{p_1p_2v}{8p_1 + 2p_2} \]

  Substitute $\lambda$ back to (\ref{eqn:c1}) and (\ref{eqn:c2}) and
  solve for $x_1$ and $x_2$:
  \[ x_1 = \left( \frac{p_2v}{8p_1 + 2p_2} \right)^2, \quad
     x_2 = \left( \frac{p_1v}{4p_1 +  p_2} \right)^2 \]
  which are the \emph{Hicksian demands} for both goods.\\

  Notice that for $v>0$ and $p>>0$, we always have $x>>0$, which implies 
  there is no corner solution for this problem. \\
  
  Therefore, the \emph{expanditure function} is solely given by
  \begin{align*}
    e(p,v) = p_1 \left( \frac{p_2v}{8p_1 + 2p_2} \right)^2 +
              p_2 \left( \frac{p_1v}{4p_1 +  p_2} \right)^2
           = \frac{p_1 p_2 v^2}{16p_1 + 4p_2}
  \end{align*}


  \item Verify that the expenditure function satisfies properties (a)-(d) of
        Proposition 10.4 in Kreps (2013).\\

  \begin{enumerate}
    \item The expenditure function is Homogeneous of degree 1. \\
    \[ e(\lambda p, v) =
        \frac{\lambda p_1\cdot \lambda p_2 v^2}{4\lambda (4p_1 + p_2)} =
        \frac{\lambda p_1 p_2 v^2}{4 (4p_1 + p_2)} =
        \lambda e(p,v)
    \]

    \item For every $p$, $e(p,v)=0$ if $v=u(0)$ and $e(p,v)>0$ if $v>u(0)$.\\

    This is obvious by observing that $u(0)=0$, and $e(p,v)=0$ if $v=0$, while
    $e(p,v)>0$ if $v>0$.\\

    \item The expenditure is strictly increasing in $v$
          and nondecreasing in $p$.\\
    \[ \frac{\partial e(p,v)}{\partial v} =
         \frac{p_1 p_2 v}{2(4p_1 + p_2)}  > 0 \quad
         \textrm{if}\ v > 0
    \]
    And, $\lim_{v \to 0^{+}}\ e(p,v) = 0$. So $e(p,v)$ is strictly increasing
    in $v$ for $v \geq 0$.

    \[  \frac{\partial e(p,v)}{\partial p_1} =
           \left( \frac{p_2v}{8p_1 + 2p_2} \right)^2 \geq 0
    \]
    \[  \frac{\partial e(p,v)}{\partial p_2} =
            \left( \frac{p_1v}{4p_1 +  p_2} \right)^2 \geq 0
    \]
    Therefore, $e(p,v)$ is nondecreasing in $p$. \\

    \item The expenditure function $e(p,v)$ is concave in $p$.\\

    The Hessian matrix of $e(p,v)$ with respect to $p$ is
    \[  H = \frac{2v^2}{(4p_1 + p_2)^3}
          \begin{bmatrix}
            -p_2^2 & p_1p_2 \\
            p_1p_2 & -p_1^2
          \end{bmatrix}
    \]
    Verify all its principal minors,
    \[ |-p_2^2| \leq 0, \quad |-p_1^2| \leq 0,\quad
       \begin{vmatrix}
         -p_2^2 & p_1p_2 \\
         p_1p_2 & -p_1^2
       \end{vmatrix} = 0 \geq 0
    \]
    So the Hessian matrix $H$ is negative semidefinite.
    Hence, the expenditure function $e(p,v)$ is concave in $p$.\\

  \end{enumerate}

\end{enumerate}

\end{document}
