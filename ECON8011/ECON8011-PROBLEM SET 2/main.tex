\documentclass{article}
\usepackage{amssymb}
\usepackage{amsthm}
\usepackage{amsmath}
\usepackage{graphicx}
\usepackage{enumerate}  
\usepackage{array,multirow}
\usepackage[usenames, dvipsnames]{xcolor}

\title{PROBLEM SET 2}
% \author{ 
%     \textsc{Zeming Wang} 
%     \thanks{u6114134@anu.edu.au}
% }
\date{}

\setlength\parindent{0pt}

\begin{document}

\maketitle

\section*{Question 1}

The alternatives the consumer faced and the choice s/he made can be summarized in the table below:
    
\begin{table}[h]
    \centering
    \begin{tabular}{ccccc}
    \hline\hline
    \multirow{2}{*}{Price} & \multirow{2}{*}{Choice}  & \multicolumn{3}{c}{Alternatives} \\\cline{3-5}
                          &                         & $(1,2,2)$ & $(2,1,2)$ & $(2,2,1)$ \\
    \hline
    $p_1=(2,1,2)$         & $x_1=(1,2,2)$           & \textcolor{red}{8} & \textcolor{gray}{9} & 8 \\
    $p_2=(2,2,1)$         & $x_2=(2,1,2)$           & 8 & \textcolor{red}{8} & \textcolor{gray}{9} \\
    $p_3=(1,2,2)$         & $x_3=(2,2,1)$           & \textcolor{gray}{9} & 8 & \textcolor{red}{8} \\
    \hline
    \end{tabular}
    % \caption{Caption}
    % \label{tab:my_label}
\end{table}

The unfeasible bundles are grayed out since the consumer's wealth $y=8$; 
and the consumer's choice is highlighted red. \\

We can verify the weak axiom of revealed preference (WARP) against the three price vectors:
{\small
\begin{enumerate}[(i)]
    \item For $p_1=(2,1,2)$, bundle $x_1=(1,2,2)$ is chosen; bundle $x_3=(2,2,1)$ is also feasible; 
          but in the situation $x_3=(2,2,1)$ is chosen, namely $p_3=(1,2,2)$, bundle  $x_1=(1,2,2)$
          is not feasible;
    \item For $p_2=(2,2,1)$, bundle $x_2=(2,1,2)$ is chosen; bundle $x_1=(1,2,2)$ is also feasible; 
          but in the situation $x_1=(1,2,2)$ is chosen, namely $p_1=(2,1,2)$, bundle $x_2=(2,1,2)$
          is not feasible;
    \item For $p_3=(1,2,2)$, bundle $x_3=(2,2,1)$ is chosen; bundle $x_2=(2,1,2)$ is also feasible;
          but in the situation $x_2=(2,1,2)$ is chosen, namely $p_2=(2,2,1)$, bundle $x_3=(2,2,1)$
          is not feasible.
\end{enumerate}
}
Therefore the weak axiom of revealed preference (WARP) is satisfied. \\

For the revealed preferences to be compatible with the generalized axiom of revealed preference (GARP),
the preferences must satisfy: $$x_1 \sim x_2 \sim x_3$$

To show this, recall the consumer's choice at different price vectors:
\begin{align*}
    \textrm{(i)} & \implies x_1 \succeq x_3 \\
    \textrm{(ii)} & \implies x_2 \succeq x_1 \\
    \textrm{(iii)} & \implies x_3 \succeq x_2
\end{align*}
Therefore we have: $$ x_1 \succeq x_3 \succeq x_2 \succeq x_1 $$
If any of the two relations hold strict, we end up with a strict revealed preference cycle, 
which violates the GARP. To maintain GARP non-violated, it is only possible when $x_1 \sim x_2 \sim x_3$.\\

\section*{Question 2}

\begin{enumerate}
\item Allais paradox violates the Independence Axiom \\

In Allais paradox, there are four lotteries:
\[
l_1 = 
\begin{cases}
    \$55,000\quad & \textrm{with prob. } 0.33 \\
    \$48,000\quad & \textrm{with prob. } 0.66 \\
    \$0\quad & \textrm{with prob. } 0.01
\end{cases} 
\qquad
l_2 = \$48,000 \quad \textrm{with prob. 1.00}
\]
\[
l_3 = 
\begin{cases}
    \$55,000\quad & \textrm{with prob. } 0.33 \\
    \$0 \quad & \textrm{with prob. } 0.67
\end{cases} 
\qquad
l_4 = 
\begin{cases}
    \$48,000 \quad & \textrm{with prob. } 0.34 \\
    \$0 \quad & \textrm{with prob. } 0.66
\end{cases}
\]
The revealed preferences are $l_2 \succ l_1$ and $l_3 \succ l_4$.\\

To show this violates the Inpendence Axiom,
Let $\delta_{55}$ denotes the probability distribution that assigns probably $1$ to outcome $\$55,000$;
$\delta_{48}$ denotes the probability distribution that assigns probably $1$ to outcome $\$48,000$;
$\delta_{0}$ denotes the probability distribution that assigns probably $1$ to outcome $\$0$.
Then $l_1, l_2, l_3, l_4$ can be treated as compound lotteries of $\delta_{55}, \delta_{48}, \delta_0$.
\begin{align*}
    l_1 &= 0.33 \delta_{55} + 0.66 \delta_{48} + 0.01 \delta_{0} \\
    l_2 &= 1.00 \delta_{48} \\
    l_3 &= 0.33 \delta_{55} + 0.67 \delta_{0} \\
    l_4 &= 0.34 \delta_{48} + 0.66 \delta_{0}
\end{align*}
Therefore,
\begin{align*}
    l_2 \succ l_1 &\iff  \delta_{48} \succ 0.33 \delta_{55} + 0.66 \delta_{48} + 0.01 \delta_{0} \\
                  &\iff 0.66\delta_{48} + 0.34\delta_{48} \succ 0.33 \delta_{55} + 0.66 \delta_{48} + 0.01 \delta_{0}
\end{align*}
Applying the Independence Axiom, $\delta_{48}$ can be replaced by $\delta_{0}$:
$$ 0.66\delta_{0} + 0.34\delta_{48} \succ 0.33 \delta_{55} + 0.67 \delta_{0} $$

But this implies $l_4 \succ l_3 $ which contradicts with the revealed preference. \\


\item Ellsberg paradox violates the Sure-Thing Principle \\

In Ellsberg paradox, the state space is $$S = \{ red, blue, green \}$$
Define subspace $T \subset S$ as $$T=\{ red\ \textrm{or}\ blue \};\ T^C=\{ green \}$$
There are four acts:
$$ b_1 = 
    \begin{cases}
        \$100 \quad & \textrm{if the ball is red} \\
        \$0   \quad & \textrm{otherwise}
    \end{cases} \qquad
    b_2 = 
     \begin{cases}
         \$100 \quad & \textrm{if the ball is blue} \\
         \$0   \quad & \textrm{otherwise}
     \end{cases}
$$
$$ b_3 = 
    \begin{cases}
        \$0   \quad & \textrm{if the ball is blue} \\
        \$100 \quad & \textrm{otherwise}
    \end{cases} \qquad
    b_4 = 
     \begin{cases}
         \$0   \quad & \textrm{if the ball is red} \\
         \$100 \quad & \textrm{otherwise}
     \end{cases}
$$

If $s \in T^C$, i.e. the ball is green,
$$ b_1(green) = b_2(green) = \$0 $$
$$ b_3(green) = b_4(green) = \$100 $$

If $s \in T$, the ball is either red or blue. If the ball is red, 
$$ b_1(red) = b_3(red) = \$100 $$
$$ b_2(red) = b_4(red) = \$0 $$ 

If the ball is blue, 
$$ b_1(blue) = b_3(blue) = \$0 $$
$$ b_2(blue) = b_4(blue) = \$100 $$ 

Therefore, $b_1 = b_3, b_2 = b_4$ whenever $s \in T$; 
$b_1 = b_2, b_3 = b_4$ whenever $s \in T^C$.\\

According to the Sure Thing Principle, we should have
$$ b_1 \succ b_2 \iff b_3 \succ b_4 $$

However, this contradicts with the revealed preferences. 
So the revealed preferences in this case violate the Sure Thing Principle. \\

\end{enumerate}

\end{document}
