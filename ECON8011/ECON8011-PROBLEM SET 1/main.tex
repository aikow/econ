\documentclass{article}
\usepackage{amssymb}
\usepackage{amsthm}
\usepackage{graphicx}

\title{PROBLEM SET 1}
% \author{ 
%     \textsc{Zeming Wang} 
%     \thanks{u6114134@anu.edu.au}
% }
\date{}

\setlength\parindent{0pt}

\begin{document}

\maketitle

\section*{Question 1}
Prove that if $X$ is a finite set and $\succeq$ is complete and 
transitive, then the function $u: X \to \mathbb{R}$ defined by 
$u(x) = \#\textrm{NBT}(x)$ satisfies $u(x) \geq u(y) \Leftrightarrow x \succeq y$.

\begin{proof} $\quad$

\begin{itemize}
\item Necessity. Suppose $x \succeq y$. Then we have 
$\textrm{NBT}(y) \subseteq \textrm{NBT}(x)$ (by Proposition 9).
Since $X$ is finite, $\textrm{NBT}(x)$ and $\textrm{NBT}(y)$ are both countable, it must follow that 
$\#\textrm{NBT}(y) \leq \#\textrm{NBT}(x)$. 
It means, by definition, $u(y) \leq u(x)$.

\item Sufficiency. Suppose $u(x) \geq u(y)$. By definition, 
$\#\textrm{NBT}(y) \leq \#\textrm{NBT}(x)$. 
Since NBT sets nest, it suggests that it must be the case 
$\textrm{NBT}(y) \subseteq \textrm{NBT}(x)$, 
which implies $x \succeq y$ (Proposition 9).
\end{itemize}    
\end{proof}


\section*{Question 2}
Suppose that $u(\cdot)$ is a continuous utility function 
representing $\succeq$. Show that $\succeq$ is complete, 
transitive, and continuous. 

\begin{proof}
Since preference $\succeq$ can be represented by a utility 
function $u: X \to \mathbb{R}$. By Proposition 5, the preference 
is complete and transitive. So we only have to prove that 
the continuity of the utility function suggests the continuity 
of the preference relation. We try to prove this by proving that 
if $u$ is continuous, then the set $\textrm{SBT}(x)$ 
is an open set. \\

By definition, $x \succeq y \Leftrightarrow u(x) \geq u(y)$. 
Therefore we can define $\textrm{SBT}(x)$ in terms of the 
utility function: 
$\textrm{SBT}(x) = \{y \in X: y \succ x\} 
                 = \{ y \in X: u(y) > u(x) \}$\\

Note that any set like $\{z \in \mathbb{R}: z > a \}$ is open.
So the set $U = \{ u: u > u_x \}$ is open (where $u_x=u(x))$. 
That means, for any $u \in U$, there exists $\epsilon>0$ 
such that for any $u'$ with $|u'-u|<\epsilon$, we have $u' \in U$. \\

While the continuity of $u(\cdot)$ suggests, given that $\epsilon$, 
there must exist an $\delta>0$ such that whenever 
$|y'-y|<\delta$, we have $|u(y')-u(y)|<\epsilon$, 
which implies $u(y') \in U$.\\

So for any $y \in \textrm{SBT}(x)$, there exists an $\delta$, whenever $|y'-y|<\delta$, we have $u(y') \in U$ and hence 
$y' \in SBT(x)$. That proves $\textrm{SBT}(x)$ is an open set,
which, by Proposition 1.14, is equivalent to that 
the preference $\succeq$ is continuous.

\end{proof}

\end{document}
