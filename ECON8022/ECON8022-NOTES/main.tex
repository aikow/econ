\documentclass[10pt,a4]{article}

\usepackage{fullpage}
\usepackage{setspace}
\usepackage{parskip}
\usepackage{titlesec}
\usepackage{xcolor}
\usepackage{lineno}





\PassOptionsToPackage{hyphens}{url}
\usepackage[colorlinks = true,
            linkcolor = blue,
            urlcolor  = blue,
            citecolor = blue,
            anchorcolor = blue]{hyperref}
\usepackage{etoolbox}
\makeatletter
\patchcmd\@combinedblfloats{\box\@outputbox}{\unvbox\@outputbox}{}{%
  \errmessage{\noexpand\@combinedblfloats could not be patched}%
}%
\makeatother


\usepackage[round]{natbib}
\let\cite\citep




\renewenvironment{abstract}
  {{\bfseries\noindent{\abstractname}\par\nobreak}\footnotesize}
  {\bigskip}

\renewenvironment{quote}
  {\begin{tabular}{|p{13cm}}}
  {\end{tabular}}

\titlespacing{\section}{0pt}{*3}{*1}
\titlespacing{\subsection}{0pt}{*2}{*0.5}
\titlespacing{\subsubsection}{0pt}{*1.5}{0pt}


\usepackage{authblk}


\usepackage{graphicx}
\usepackage[space]{grffile}
\usepackage{latexsym}
\usepackage{textcomp}
\usepackage{longtable}
\usepackage{tabulary}
\usepackage{booktabs,array,multirow}
\usepackage{amsfonts,amsmath,amssymb}
\providecommand\citet{\cite}
\providecommand\citep{\cite}
\providecommand\citealt{\cite}
% You can conditionalize code for latexml or normal latex using this.
\newif\iflatexml\latexmlfalse
\providecommand{\tightlist}{\setlength{\itemsep}{0pt}\setlength{\parskip}{0pt}}%

\AtBeginDocument{\DeclareGraphicsExtensions{.pdf,.PDF,.eps,.EPS,.png,.PNG,.tif,.TIF,.jpg,.JPG,.jpeg,.JPEG}}

\usepackage[utf8]{inputenc}
\usepackage[greek,english]{babel}






% Use this header.tex file for:
% 1. frontmatter/preamble LaTeX definitions
%  - Example: \usepackage{xspace}
% 2. global macros available in all document blocks
%  - Example: \def\example{This is an example macro.}
%
% You should ONLY add such definitions in this header.tex space,
%  and treat the main article content as the body/mainmatter of your document
% Preamble-only macros such as \documentclass and \usepackage are
% NOT allowed in the main document, and definitions will be local to the current block.
\usepackage{amssymb, amsmath, amsthm, latexsym}



\title{Class Notes on Macroeconomic Theory}

\author[1]{Zeming Wang}%
\affil[1]{Australian National University}%
\date{}




\begin{document}

\maketitle
 
\tableofcontents

% \begingroup
% \let\center\flushleft
% \let\endcenter\endflushleft
% \maketitle
% \endgroup

\newpage






\section{Dynamic Programming}

{\label{895037}}

Consider a typical problem (social planner's problem): a social planner
plans for the economy how much to consume and invest for each period in
time to maximize the lifetime utility. Mathematically, the social
planner wants to solve the optimisation problem:

~

$$ \max_{\{c_t\}} \sum_{t=0}^{\infty} \beta^t U(c_t) $$

subject to the budget constraint:
$$ c_t + i_t = f(k_t) $$

and the law of motion of capital:
$$ k_{t+1} = (1-\delta)k_t + i_t $$



Assume $k_0$ is given. $\beta$ the discount factor satisfies $0<\beta<1$. 

We also assume, throughout this note, the utility function $U$ satisfies the following properties:
\begin{itemize}
\item $U$ is a twice continuously differentiable real-valued function;
\item $U$ is strictly increasing and strictly concave;
\item Inada condition: $\lim_{c\to 0} U'(c) = \infty$, $\lim_{c\to\infty} U'(c)=0$.
\end{itemize}

The production technology $f$ satisfies the properties:
\begin{itemize}
\item $f$ is twice continuously differentiable;
\item $f$ is homogeneous of degree one in its input;
\item The output of $f$ is increasing in each input at a decreasing rate;
\item Inada condition: $\lim_{k\to 0} f'(k) = \infty$, $\lim_{k\to\infty} f'(k)=0$.
\end{itemize}

For simplicity, in this notes, we assume capital depreciates fully by the end of each period, i.e. $\delta=1$. And we substitute the budget constraint into the objective function to avoid solving optimisation problem with constraints. Therefore, the problem can be reduced to:

$$ \max_{\{k_{t+1}\}} \sum_{t=0}^{\infty} \beta^t U(f(k_t) - k_{t+1}) $$



How do we solve a sequence of $\{k_{t+1}\}_{t=0}^{\infty}$? In paticular, how do we maximize over infinite time horizon? To shed some light on the technique, we begin with somewhat more approachable finite horizon prblem, and extend it to infinite horizon. 

\subsection{Finite Horizon}

{\label{609724}}\par\null

Suppose the time ends at $T$, we want to solve:

$$ \max_{\{k_{t+1}\}} \sum_{t=0}^{T} \beta^t U(f(k_t) - k_{t+1}) $$

\subsubsection{Solve by brutal force}

{\label{571029}}\par\null

Set up a Lagrangian:

$$ \mathcal{L} = \sum_{t=0}^{T} \beta^t U(f(k_t) - k_{t+1}) $$

The first-order condition (with respect to $k_{t+1}$) is:

$$ -\beta^t U'(c_t) + \beta^{t+1} U'(c_{t+1}) f'(k_{t+1}) = 0 $$

Rearrange it a little bit:

$$ \underbrace{\frac{U'(c_t)}{\beta U'(c_{t+1})}}_{\textrm{MRS: marginal rate of substitution}} = 
   \underbrace{f'(k_{t+1})}_{\textrm{MRT: marginal rate of transformation}}
$$

This intertemporal first-order condition is called the \textbf{Euler equation}.



Additionally, a boundary condition needs to be satisfied:

$$ k_{T+1} = 0 $$

There are $T+1$ unknowns $\{k_{t+1}\}_{t=0}^{T}$ and $T+1$ equations, we should be able to solve all $\{k_{t+1}\}_{t=0}^{T}$.

\subsubsection{Discover the recursive nature of the
problem}

{\label{336406}}

If we can discover the recursive nature of the problem, we can hopefully simplify it. Let $V_T(k_0)$ be the \textbf{value function}.
\[
    \begin{aligned}
        V_T(k_0) &= \max \sum_{t=0}^T \beta^t U(f(k_t)-k_{t+1}) \\
                 &= \max \{U(f(k_0)-k_1) + \beta \sum_{t=0}^{T-1} \beta^t U(f(k_t)-k_{t+1})\}  \\ 
                 &= \max \{U(f(k_0)-k_1) + \beta V_{T-1}(k_1)\}
    \end{aligned}
\]
If $V_{T-1}$ is known, we will be able to solve it just like solving one-choice-variale optimisation!



The clue is from the boundary condition. Since $k_{T+1} = 0$, $V_{-1}(k_{T+1})=0$ is known. Then we will be able to solve backward: $V_0, V_1,..., V_T$.

$$ V_0(k_T) = \max_{k_{T+1}} \{U(f(k_T)-k_{T+1}) + \beta \underbrace{V_{-1}(k_{T+1})}_{=0}\}$$
The optimal choice is of course $ k_{T+1} = 0 $, and the value function becomes $ V_0(k_T) = U(f(k_T)) $.

Now $V_0(k_T)$ is known, we can use it to solve $V_1(k_{T-1})$:
$$ V_1(k_{T-1}) = \max_{k_{T}} \{U(f(k_{T-1})-k_{T}) + \beta \underbrace{V_{0}(k_{T})}_{\textrm{known}}\}$$
Solve this maximization problem, we we will get 
$$ k_{T} = g_{T-1}(k_{T-1}) $$
i.e. choice variable $k_T$ as a function of the given state $k_{T-1}$. $g_{T-1}$ is called the \textbf{policy function} at date $T-1$. It tells us the optimal choice at $T-1$.
Substitute in $ k_{T} = g_{T-1}(k_{T-1}) $, we value function is given by
$$V_1(k_{T-1}) = U(f(k_{T-1})-g_{T-1}(k_{T-1})) + \beta V_0(k_{T}) $$

Now $V_1(k_{T-1})$ is known, we can continue to solve $V_2(k_{T-2})$.
$$\vdots$$

Keep going with this process, we will solve a sequence of functions: $\{V_{-1}, V_0, V_1,..., V_T\}$ and $\{g_T, g_{T-1}, ... , g_0\}$, which are the solutions to the finite horizon problem.

\subsection{Infinite Horizon}

{\label{327193}}

Exploring the finite horizon problem enlightens us to solve the the
infinite horizon problem recursively. It turns out solving the infinite
horizon problem recursively is even easier than the finite horizon
problem.~

It can be proved, as $T\to\infty$, we have $V_T \to V$, i.e. the value function will have the same form for all period $t$. 


We can thus write the recursive optimization problem as:

$$ V(k) = \max_{k'} \{ U(f(k)-k') + \beta V(k') \} $$

This is called the \textbf{Bellman equation}. The solution to the Bellman equation is the value function $V$.



To solve the value function, define
$$ (TV)(k) = \max_{k'} \{ U(f(k)-k') + \beta V(k') \} $$

$T$ is a function that maps a given function $V$ to a new $V$. If we can find the fixed point of $T$, i.e. $(TV)(k) = V(k)$, then it is the solution to our Bellman equation.

But how do we know such a fixed point exists, and how can we find it? To
answer these questions, we need a small technical detour on Banach fixed
point theorem.

\par\null

\subsubsection{Banach fixed point
theorem}

{\label{335214}}\par\null

\textbf{Banach fixed point theorem}: If $(S,d)$ is a \emph{complete} metric space and $T:S \to S$ is a \emph{contraction}, then there is a unique fixed point for $T$.



\textbf{Blackwell's sufficient condition for a contraction}: Let $M: B(X) \to B(X)$ be any map satisfying
\begin{itemize}
\item Monotonicity: For any $v,w \in B(X)$ such that $w \ge v \implies Mw \ge Mv$.
\item Discounting: There exists a $0 \le \selectlanguage{greek}β \selectlanguage{english}< 1$ such that $M(w+c)=Mw+\selectlanguage{greek}β\selectlanguage{english}c$, for all $w \in B(X)$ and $c \in \mathbb{R}$. (Define $(f+c)(x)=f(x)+c$)
\end{itemize}
Then $M$ is a contraction with modulus $\beta$.

\subsubsection{Fixed point for the infinite horizon
problem}

{\label{142845}}

Given the assumptions on $U$ and $f$, the value function is continuous, and the state space $S \ni k$ is  bounded (there is an upper bound for capital accumulation). $T: C_b(S) \to C_b(S)$ is a mapping from the space of continuous bounded functions ($C_b(S)$) to itself. It can be proved the space $C_b(S)$ is complete when equipped with the uniform metric $d_u(v,w) = \sup_{x\in S}|v(x) - w(x)|$. 

The Blackwell's sufficient condition is also satisfied:
\begin{itemize}
\item Monotonicity: let $v,w \in C_b(S)$ and $w \ge v$, 
      $$ Tw = \max\{U + \beta w\} \ge \max\{U + \beta v\} = Tv $$
\item Discounting: let $w \in C_b(S)$ and let $c \in \mathbb{R}$, 
      $$ T(w+c) = \max\{U + \beta(w+c)\} = \max\{U + \beta w\} + \beta c = Tw + \beta c $$
\end{itemize}

Therefore, $T$ is contraction on a complete metric space. By Banach fixed point theorem, $T$ has a unique fixed point. 

\subsubsection{Solving the Bellman
equation}

{\label{838588}}

The contraction mapping argument also provides us a mechanical way to
solve the fixed point for the Bellman equation. We can iterate on an
arbitrary initial guess for the value function, even the guess is
incorrect, the iteration will eventually converge to the true value
function. This is the way how computers solve the problem. In terms of
solving the Bellman equation by hand, we usually guess the value
function (or policy function) to be a particular form, and solve for the
parameters. Specifically, there are mainly two approaches: value
function approach and policy function approach.

\par\null

To demostrate these two approaches, in the following text, we specialize the utility function and the production function to the following form:
$$ U(c) = \ln (c) $$
$$ f(k) = k^{\alpha} $$

\par\null

\textbf{Value function (Bellman operator) approach:}

\par\null

Guess $V(k) = A + B \ln(k)$. Substitute this form into the Bellman equation:
$$ A + B \ln(k) = \max_{k'} \{\ln(k^{\alpha} - k') + \beta(A + B \ln(k'))\}$$

Solve for $k'$: 
$$k' = \frac{\beta B}{1 + \beta B} k^{\alpha}$$

Substitute $k'$ into the value function:
$$ A + B \ln(k) = \ln(k^{\alpha} - \frac{\beta B}{1 + \beta B} k^{\alpha}) + \beta(A + B \ln(\frac{\beta B}{1 + \beta B} k^{\alpha})) $$

Rearrange the equation:
$$ A + B\ln(k) = \beta A + \ln(\frac{1}{1+\beta B}) + \beta B \ln(\frac{\beta B}{1+\beta B}) + (1+\beta B)\alpha \ln(k) $$

To make LHS = RHS, we require
$$ B = (1+\beta B)\alpha $$
$$ A = \beta A + \ln(\frac{1}{1+\beta B}) + \beta B \ln(\frac{\beta B}{1+\beta B}) $$
Therefore, 
$$ B = \frac{\alpha}{1-\alpha\beta}$$
$$ A = \frac{1}{1-\beta}\left[\ln(1-\alpha\beta)+\frac{\alpha\beta}{1-\alpha\beta}\ln(\alpha\beta)\right]$$
Therefore, 
$$ k' = g(k) = \alpha\beta k^{\alpha} $$
$$ V(k) = \frac{1}{1-\beta}\left[\ln(1-\alpha\beta)+\frac{\alpha\beta}{1-\alpha\beta}\ln(\alpha\beta)\right] + \frac{\alpha}{1-\alpha\beta}\ln(k)$$

\textbf{Policy function (Euler operator) approach}

\par\null

Guess $k' = g(k) = \theta k^{\alpha}$, then $c = (1-\theta)k^{\alpha}$. Substitute this into the Euler equation:
$$ \frac{1}{(1-\theta)k^{\alpha}} = \alpha\beta\frac{1}{(1-\theta)k'^{\alpha}}k'^{\alpha-1} = \alpha\beta\frac{1}{(1-\theta)k'}$$
$$ \frac{1}{(1-\theta)k^{\alpha}} = \alpha\beta\frac{1}{(1-\theta)\theta k^{\alpha}}$$
$$\frac{1}{1-\theta} = \alpha\beta\frac{1}{\theta(1-\theta)}$$
Therefore, $\theta = \alpha\beta$.



In both approaches, we can see the optimal savings rate is a fixed proportion of the total income. The fixed proportion $\alpha\beta$ is called the \textbf{Golden rule} of savings.

\newpage

\section{Deterministic Competitive
Equilibrium}

{\label{984904}}

In stead of modelling the economy from a social planner's perspective,
it is more realistic to model the economy in a competitive equilibrium
setting, in which households maximise their utilities and firms maximise
their profits. In the following context, we assume homogeneous
households and firms, i.e. the economy can be represented by only one
household and one firm that behave competitively to each other.~

\subsection{Arrow-Debreu Equilibrium}

{\label{623891}}

In an Arrow-Debreu setting, there is a centralised market that opens
only at the very beginning of time. In this market, agents contract on
how much to transfer to each other in all dates in the future, once and
for all. All future plans and transfers are then determined. Agents
trade to trade to maximise their lifetime utilities and profits.~

~

Let $q_t^0$ be the relative price of a date-$t$ claim to (consumption or investment) goods in units of date-$0$ goods. 

\textbf{Household's problem:}

$$ \max_{\{c_t,n_t,i_t\}} \sum_{t=0}^{\infty}\beta^t U(c_t, n_t) - \mu\underbrace{\left[\sum_{t=0}^{\infty}q_t^0(c_t+i_t) - \sum_{t=0}^{\infty}q_t^0(r_tk_t + w_tn_t) - \Pi_0\right]}_{\textrm{lifetime budget constraint}} $$

\begin{center}subject to $k_{t+1} = (1-\delta)k_t + i_t$\end{center}



The first-order conditions are:
\[
    \begin{aligned}
        c_t &: \beta^t U_c - \mu q_t^0 = 0 \\
        n_t &: \beta^t U_n + \mu q_t^0 w_t = 0 \\
        k_{t+1} &: q_t^0 - q_{t+1}^0 (1-\delta) - q_{t+1}^0 r_{t+1} = 0
    \end{aligned}
\]

We can further derive:
$$ -\frac{U_n(c_t, n_t)}{U_c(c_t, n_t)} = w_t $$
$$ \frac{U_c(c_t, n_t)}{\beta U_c(c_{t+1}, n_{t+1})} = \frac{q_t^0}{q_{t+1}^0} $$
$$ \frac{q_t^0}{q_{t+1}^0} = 1-\delta + r_{t+1} $$

These conditions says: wage is the price for leisure in terms of
consumption; the relative price cross two periods is equal to the
marginal rate of substitution of consumption between the periods and the
gross return through the time.~

\par\null

\textbf{Firm's problem:}

\par\null

$$\Pi_0 = \max_{\{K_t,N_t\}} \sum_{t=0}^{\infty} q_t^0 [F(K_t, N_t) - r_tK_t - w_tN_t]$$



The first-order condition gives:
$$ F_K(K_t, N_t) = r_t $$
$$ F_N(K_t, N_t) = w_t $$

\textbf{Arrow-Debreu equilibrium}

\textbf{}

An Arrow-Debreu equilibrium is a date-contingent allocation $\{C_t, N_t, K_{t+1}\}$ and prices $\{q_t^0, q_t^0w_t, q_t^0r_t\}$ such that 
\begin{itemize}
\item Given prices, $\{c_t, n_t, k_{t+1}\}$ is optimal for the household;
\item Given prices, $\{K_t, N_t\}$ is optimal for the firm;
\item Market clears for all dates: $c_t = C_t, k_t=K_t, i_t=I_t, n_t=N_t$; $K_{t+1}=(1-\delta)K_t + I_t$; $C_t+I_t=F(K_t,N_t)$.
\end{itemize}

\subsection{Sequential Competitive
Equilibrium}

{\label{273711}}

In a sequential competitive equilibrium modelling, households and firms
maximise trade for each period, therefore the budget constraint has to
hold for every period (instead of a single lifetime budge constraint).
We assume households own all the capital, and firms pay rent to use the
capital. There is an alternative model in which firms own capital and
households own shares of the firms, but we skip this model in this
note.~

\par\null

\textbf{Household's problem}

\textbf{}

$$\max \sum_{t=0}^{\infty}\left[\beta^t U(c_t, n_t) - \mu_t(c_t + i_t - r_tk_t - w_tn_t - \pi_t) \right]$$
$$\textrm{subject to } k_{t+1}=(1-\delta)k_t + i_t$$



The first-order conditions are:
\[
    \begin{aligned}
        c_t &: \beta^t U_c - \mu \mu_t^0 = 0 \\
        n_t &: \beta^t U_n + \mu \mu_t^0 w_t = 0 \\
        k_{t+1} &: \mu_t^0 - \mu_{t+1}^0 (1-\delta) - \mu_{t+1}^0 r_{t+1} = 0
    \end{aligned}
\]
Therefore we have:
$$ -\frac{U_n(c_t, n_t)}{U_c(c_t, n_t)} = w_t $$
$$ \frac{U_c(c_t, n_t)}{\beta U_c(c_{t+1}, n_{t+1})} = \frac{\mu_t^0}{\mu_{t+1}^0} $$
$$ \frac{\mu_t^0}{\mu_{t+1}^0} = 1-\delta + r_{t+1} $$

\textbf{Firm's problem}

\textbf{}

The firm maximizes its profit for each period $t$:
$$\pi_t = \max_{\{K_t,N_t\}} [F(K_t, N_t) - r_tK_t - w_tN_t]$$

The first-order conditions are as usual:
$$ F_K(K_t, N_t) = r_t $$
$$ F_N(K_t, N_t) = w_t $$

\textbf{Sequential competitive equilibrium}

\textbf{}

A sequential competitive equilibrium is an allocation $\{C_t, N_t, K_{t+1}\}$ and prices $\{w_t, r_t\}$ such that 
\begin{itemize}
\item Given prices, the plan $\{c_t, n_t, k_{t+1}\}$ is optimal for the household;
\item Given prices, $\{N_t, K_t\}$ maximize the firm's prifit;
\item Market clears for every period $t$.
\end{itemize}

We can see the equilibrium system in a sequential competitive setting is in effect identical to that in the Arrow-Debreu setting. $\frac{q_t^0}{q_{t+1}^0}$ is just a special case for $\frac{\mu_t}{\mu_{t+1}}$.

We can also find that the equilibrium allocations implied by a
competitive market are identical to the social planner's allocations,
which means the competitive equilibrium allocations are Pareto efficient
(First Welfare theorem). Conversely, we can also find equilibrium prices
to support a Pareto optimal allocation for different market arrangement
(Second Welfare theorem).~

\par\null

\subsection{Recursive Competitive
Equilibrium}

{\label{793453}}\par\null

If the equilibrium decision rules are time invariant, we can describe a competitive equilibrium in a recursive fashion, in which the households maximise:
$$ V(k,K) = \max_{c,n,k'} \{ U(c,n) + \beta V(k',K') \}$$

A recursive competitive equilibrium in an economy is 
\begin{itemize}
    \item a value function: $V(k,K)$
    \item decision functions: $c = g^c(k,K)$, $n = g^n(k,K)$, $k' = g^k(k,K)$
    \item prices: $r(K)$, $w(K)$
    \item law of motion: $K' = G(K)$
\end{itemize}
such that 
\begin{itemize}
    \item given prices, $g^c,g^n,g^k$ solve the household's Bellman equation;
    \item $r(K)$ and $w(K)$ satisfy the firm's optimization condition;
    \item market clears: $k=K$, $g^k(k,K) = G(K)$, $g^c(k,K)=C$, $g^n(k,K)=N$, $K'-(1-\delta)K + C=F(K,N)$.
\end{itemize}

Solving for a recursive competitive equilibrium involves more work than
solving for a sequential equilibrium, since it involves solving for the
value function and policy functions, which specify not only equilibrium
behaviors but also ``off-equilibrium'' behaviors: what the agents could
do if he deviates from the representative agent.


\newpage 

\section{Fiscal Policy and Optimal
Taxation~}

{\label{229543}}\par\null

Suppose a tax $\tau_t$ is levied on households' income (rent and wage) in each period $t$. Suppose tax is collected to form public capital $G$ in production. Assume labor supply $L_t=1$, and $\{\tau_t\}_{t=0}^{\infty}$ are exogenously given. We specialize the utility function
$$ U(c) = \ln c $$ 
and the production function
$$ F(K,G) = AG^{\theta}K^{\alpha}$$



The household's problem becomes:
$$ \max_{\{c_t, k_{t+1}\}} \left\{ \sum_{t=0}^{\infty} \beta^t\ln(c_t): c_t + k_{t+1} = (1-\tau_t)(w_t + r_tk_t + \pi_t) \right\}$$

The firm's problem is as usual:
$$ \pi_t = \max \{F(K_t, G_t) - r_t K_t - w_t \}$$

What is the recursive competitive equilibrium in the presence of
taxation?~

\par\null

\subsection{Optimal Savings under
Taxation~~}

{\label{553361}}\par\null

Rewrite the household's problem to a Bellman equation:

$$V^h(k_t, G_t) = \max_{c_t,k_{t+1}} \{\ln(c_t) + \beta V^h(k_{t+1}): c_t + k_{t+1} = (1-\tau_t)F(k_t,G_t)\}$$



Guess the solution having form $V^h(k,G)=A + B\ln(k)+C\ln(G)$. 
Solving the Bellman equation gives the policy functions:
$$ k_{t+1} = (1-\tau_t)(\alpha\beta)y_t $$
$$ c_t = (1-\tau_t)(1-\alpha\beta)y_t $$
where $y_t = AG_t^{\theta}k_t^{\alpha}$.

Therefore the optimal decision is still to save a fixed proportion
(\(\alpha\beta\)) of after-tax income.

\par\null

\subsection{Optimal Taxation}

{\label{810937}}

What is the optimal tax rates~\(\{\tau_t\}_{t=0}^{\infty}\) in an economy where
agents best respond according to their competitive equilibrium behavior?
We model this decision problem (optimal taxation) as a government's
Bellman equation:

\par\null

$$V^G(k_t, G_t) = \max_{\tau_t} \{ \ln[\underbrace{(1-\tau_t)(1-\alpha\beta)y_t}_{\textrm{best choice of household}}] + \beta V^G(k_{t+1}, G_{t+1})\}$$
$$ \textrm{subject to } k_{t+1} = (1-\tau_t)\alpha\beta y_t, G_{t+1} = \tau_t y_t $$

Guess $V^G(k,G) = A + B\ln(k) + C\ln(G)$. 

We will solve 
$$ V^G(k,G) = A + \frac{\alpha}{1-(\alpha+\theta)\beta}\ln(k) + \frac{\theta}{1-(\alpha+\theta)\beta}\ln(G) $$

The optimal tax decision is: $\tau_t = \beta\theta$.

So the optimal tax plan is to tax private-sector income at exactly the
elasticity of private income with respect to the public capital input.

\par\null

\section{Stochastic Dynamic
Programming}

{\label{704756}}

\subsection{Stochastic Production
Technology}

{\label{991631}}

Suppose we introduce a random ``shock" to out production output:

$$ y_t = \theta_t f(k_t) $$

where $\{\theta_t\}$ is a sequence of $i.i.d.$ random variables.

The social planner's problem is now to maximize the lifetime expected utility:

$$\max\left\{ \mathbb{E}_0 \sum_{t=0}^{\infty} \beta^t U(c_t): c_t + k_{t+1} = \theta_t f(k_t) \right\}$$

where $\mathbb{E}_t$ is the expectation conditioned on information available at date $t$.



The household's Bellman equation is:

$$ V(k_t, \theta_t) = \max_{c_t, k_{t+1}} \left\{U(c_t) + \beta\mathbb{E}_t V(k_{t+1}, \theta_{t+1}): c_t + k_{t+1} = \theta_t f(k_t) \right\}$$



If we specify $f(k_t) = \theta_t k_t^{\alpha}$, $U(c_t) = \ln(c_t)$, and further assume $$\ln(\theta_{t+1}) = \rho\ln(\theta_t) + \epsilon_{t+1} $$
where $\epsilon_{t+1}\sim N(0,\sigma^2)$. 
What is the household's optimal consumption and savings plan?

Follow the usual steps to solve the Bellman equation by conjecture:
$$ V(k, \theta) = A + B\ln(k) + C\ln(\theta) $$

Substitute it into the Bellman equation:
\[
   \begin{aligned}
     T \circ V(k,\theta) &= \max \{\ln(\theta k^{\alpha} - k') + \beta\mathbb{E}[A + B\ln(k') + C\ln(\theta')|\theta]\} \\
     &= \max \{\ln(\theta k^{\alpha} - k') + \beta A + \underbrace{\beta B\ln(k')}_{k_{t+1} \textrm{ is known at date } t} + \underbrace{\beta C\rho\ln(\theta)}_{\mathbb{E}[\ln(\theta')] = \mathbb{E}[\rho\ln(\theta)+\epsilon']=\rho\ln(\theta)} \}
   \end{aligned}
\]

We will solve out the value function:
$$ V(k,\theta) = A + \frac{\alpha}{1-\alpha\beta}\ln(k) + \frac{1}{(1-\beta)(1-\beta\rho)}\ln(\theta) $$

and the policy functions:
$$
    \begin{aligned}
        k' &= \alpha\beta k^{\alpha} \theta \\
        c' &= (1-\alpha\beta) k^{\alpha} \theta \\
    \end{aligned}
$$

So the optimal savings plan is unchanged in face of stochastic
technological shocks --- save a fixed proportion of total income for
every period.~

\par\null

\subsection{Stochastic Technology with Markov
Properties}

{\label{530082}}

Suppose the technology in the production function can take only finitely
many possible states:~\(\{z_1, z_2, ..., z_n\}\). Suppose the stochastic
process~\(\{z_t\}\) is governed by a time homogeneous Markov
chain, i.e. there is a pre-determined probability~\(\mathbb{P}(z_j|z_i)\)
associated with every~\(z_i, z_j \in \{z_1, ..., z_n \}\).~

The social planner's problem becomes:

$$ V(k, z_i) = \max_{k'}\left\{U(c) + \beta\sum_{z_j}V(k', z_j)\mathbb{P}(z_j|z_i)\right\} $$

Note that for each $k$, there is a vector of value functions:
$$ \mathbf{V}(k) = (V(k,z_1), ..., V(k,z_n)) = (V_1(k), ..., V_n(k)) $$

The Bellman operator now operates on vectors:
\[
\mathbf{T}\circ \mathbf{V}(k) = 
    \begin{bmatrix}
        T_1 \circ V(k,z_1) \\
        T_2 \circ V(k,z_2) \\
        \vdots \\
        T_n \circ V(k,z_n)
    \end{bmatrix} = 
    \begin{bmatrix}
        \max\left\{ U(c) + \beta\sum_{j=1}^{n} V(k', z_j)\mathbb{P}(z_j|z_1) \right\} \\
        \max \left\{ U(c) + \beta\sum_{j=1}^{n} V(k', z_j)\mathbb{P}(z_j|z_2) \right\} \\
        \vdots \\
        \max \left\{ U(c) + \beta\sum_{j=1}^{n} V(k', z_j)\mathbb{P}(z_j|z_n) \right\}
    \end{bmatrix}
\]



Therefore, $\mathbf{T}$ is a mapping from $[C_b(K)]^n$ to itself. 
Since each component $T_i$ of $\mathbf{T}$ is a contraction mapping, $\mathbf{T}$ is also a contraction mapping. So the Banach fixed point theorem applies. We can solve for $\mathbf{T}$ iteratively as we do in single function case. 

\newpage

\section{Equilibrium in Complete
Markets}

{\label{610479}}

What does the equilibrium look like in the presence of stochastic states
in an economy? In this section, we will explore a pure exchange economy
with state-dependent endowments and consumptions in a complete market. A
\textbf{complete market} is a market where there is a price for every
asset in every possible states of the world. State-dependent endowments
for individuals generate uncertainty and risk for individuals. We will
see how a complete market secures individual risks.

\par\null

Some notations used in this section: 

\begin{itemize}
\item Stochastic events: $s_t \in S = \{s_1, ..., s_n\}$
\item History of events: $h_t = (s_0, s_1, ..., s_t)$
\item Probability of events (history): $\pi(s_t)$, $\pi(h_t)$
\item Agents: $I = \{1,2,..., m\}$
\end{itemize}

Endowment for individual $i$ at time $t$ is denoted as $y_t^i(h_t)$. 

The initial state $s_0$ is assumed to be given, i.e. $\pi(s_0)=1$.

\subsection{Social Planner's Problem}

{\label{864866}}

The social planner maximises the expected lifetime utility for all agents in the economy:

$$\max \sum_{t=0}^{\infty}\sum_{h_t}\left\{\sum_{i\in I}\lambda_i\beta^tU(c_t^i(h_t))\pi_t(h_t) + \theta_t(h_t)\sum_{i\in I}[y_t^i(h_t)-c_t^i(h_t)]\right\}$$

$\lambda_i$ is the weight that the planner assigns to each agient $i$. 

Note the Lagrangian multiplier $\theta_t(h_t)$ is also history-contingent. The constraint $$\sum_{i\in I}[y_t^i(h_t)-c_t^i(h_t)]\ge 0$$ requires that the aggregrate consumption must not exceed the aggregate endowment.



The first-order condition is:

$$ \lambda_i \beta^t U'(c_t^i(h_t)) \pi_t(h_t) - \theta_t(h_t) = 0 $$

Therefore, 
$$ \frac{U'(c_t^i(h_t))}{U'(c_t^j(h_t))} =\frac{\lambda_j}{\lambda_i} $$
$$ c_t^i(h_t) = U'^{-1}\left[\frac{\lambda_j}{\lambda_i}U'(c_t^j(h_t))\right] $$

The budge constraint holds:

$$ \sum_{i\in I} y_t^i(h_t) = \sum_{i\in I} c_t^i(h_t) $$

Sum up $c_t^i$ for all $i$:

$$ \sum_{i\in I} U'^{-1}\left[\frac{\lambda_j}{\lambda_i}U'(c_t^j(h_t))\right] = \sum_{i\in I} y_t^i(h_t) $$

Given the assumption on~\(U\), we can uniquely solve
for~\(c_t^j\). Once we solve~\(c_t^j\),
every~\(c_t^i\) (\(i\neq j\)) is uniquely determined.
Therefore, the optimal allocation is only a function of realised
aggregate endowment and does not depend on the particular
history~\(h_t\) leading to this outcome or the realisation of
individual endowment.~

\par\null

\subsection{Arrow-Debreu Equilibrium}

{\label{267939}}

Suppose the market opens only at date~\(t=0\). In the
centralised market, agents exchange claims on time-\(t\)
consumption contingent on history~\(h_t\) at
price~\(q_t^0(h_t)\). Assume there is a complete set of securities
for all possible~\(h_t\).

\par\null

For each individual $i$, s/he solve the problem:

$$ \max\left\{ \sum_{t=0}^{\infty}\sum_{h_t}\beta^tU(c_t^i(h_t))\pi_t(h_t) + \mu_i\sum_{t=0}^{\infty}\sum_{h_t}q_t^0(h_t)[y_t^i(h_t)-c_t^i(h_t)]\right\} $$

Note that trades happen at date-\(0\) once and for all
implies that each individual faces only one lifetime budge constraint.

\par\null

The first-order condition is:

$$ \beta^t U'(c_t^i(h_t))\pi_t(h_t) - \mu_i q_t^0(h_t) = 0 $$

Therefore, 

$$ \frac{U'(c_t^i(h_t))}{U'(c_t^j(h_t))} =\frac{\mu_i}{\mu_j} $$

The equilibrium condition here is actually equivalent to the first-order
condition of the social planner's problem only by
letting~\(\mu_i = \lambda_i^{-1}\). So we still have the similar result: the
Arrow-Debreu allocation is a function of the realised aggregate
endowment and does not depend on either the realisation of individual
endowment or the particular history leading up to that outcome.

\par\null

The implication is, since agents can trade contracts on all possible
contingencies in the future, the market aggregate the risks and
eliminate individual risks. Or we could say, the individual risks are
fully secured by the complete market.~

\par\null

\subsubsection{Asset pricing
implications}

{\label{200528}}

In a complete market, every contingency is priced by~\(q_t^0(h_t)\).
So any other asset is ``redundant''. Any other asset can be priced with
a combination of~\(q_t^0(h_t)\).~

~

Suppose an asset pays a stream of dividents $\{d_t(h_t)\}_{t\ge 0}$, then the asset has a price at date-$0$:

$$ p^0(h_t) = \sum_{t=0}^{\infty}\sum_{h_t}q_t^0(h_t)d_t(h_t) $$

The price of the asset at date-$t$ is given by:

$$ p^t(h_{t+k}) = \sum_{k\ge 0}\sum_{h_{t+k}|h_t}q_{t+k}^{t}(h_{t+k})d_{t+k}(h_{t+k}) $$

where $q_{t+k}^{t}(h_{t+k})$ is the relative price:

$$ q_{t+k}^{t}(h_{t+k}) = \beta^k \frac{U'(c_{t+k}^{i}(h_{t+k}))}{U'(c_t^i(h_t))}\pi_t(h_{t+k}|h_t) $$

\subsection{Sequential Market with Arrow
Securities}

{\label{302363}}

Now suppose at each date~\(t\) with
history~\(h_t\), traders meet to trade for contingent goods
deliverable at~\(t+1\).

Let~\(a_t^i\left(h_t\right)\) be the total \textbf{asset} or \textbf{wealth}
owned by individual~\(i\) at date~\(t\).

Individual $i$ solves:

$$ \max \sum_{t=0}^{\infty}\sum_{h_t}\left\{ \beta^tU(c_t^i(h_t))\pi_t(h_t) + \eta_t^i(h_t)\left[y_t^i(h_t) + a_t^i(h_t) - c_t^i(h_t) - \sum_{s_{t+1}}a_{t+1}^i(s_{t+1},h_t)Q_t(s_{t+1}|h_t)\right]\right\}  $$

where $Q_t(s_{t+1}|h_t)$ is called the intertemporal pricing kernel. 

Note that~\(a_{t+1}^i\) can be negative, i.e. agents can borrow
assets. In principle, we should restrict that each agent cannot borrow
more than the maximal amount that he could repay --- the present value
of all his future endowment:

$$-a_{t+1}^i(s_{t+1}) \le \sum_{k\ge 1}\sum_{h_{t+k}|h_{t+1}} q_{t+k}^{t+1}(h_{t+k}) y_{t+k}^i (h_{t+k}) $$

But given the Inada condition, the debt constraint will never bind. If
the constraint binds, consumption from~\(t+1\) on has to be
zero forever. But this implies that the consumer has infinite marginal
utility for future consumptions, s/he could be better off by just saving
a little from today for future consumption.

The first-order conditions are:

\[
    \begin{aligned}
        c_t^i &: \beta^t U(c_t^i(h_t))\pi_t(h_t) -\eta_t^i(h_t) = 0 \\
        a_{t+1}^i &: -\eta_t^i(h_t)Q_t(s_{t+1}|h_t) + \eta_t^i(s_{t+1}, h_t) = 0
    \end{aligned}
\]

We can solve out the pricing kernel: 

$$ Q_t(s_{t+1}|h_t) = \beta\frac{U'(c_{t+1}^i(h_{t+1}))}{U'(c_t^i(h_t))}\pi_t(h_{t+1}|h_t) $$

We will show that, with an appropriate choice of initial wealth
distribution, there is an equivalence in allocations under Arrow-Debreu
equilibrium and sequential market equilibrium.

\par\null

\subsubsection{Equivalence of
equilibriums}

{\label{484195}}

If we define wealth as the present value of all the household's owned \emph{current and future net claims on consumption} ($\Omega_t^i$), i.e.

\[
    \begin{aligned}
      a_t^i(h_t) = \Omega_t^i(h_t) &= \sum_{k\ge 0}\sum_{h_{t+k}|h_t} q_{t+k}^t(h_{t+k})[c_{t+k}^i(h_{t+k}) - y_{t+k}^i(h_{t+k})] \\
      &= \sum_{k\ge 0}\sum_{h_{t+k}|h_t} \beta^k \frac{U'(c_{t+k}^i(h_{t+k}))}{U'(c_t^i(h_t))}\pi_t(h_{t+k}|h_t)[c_{t+k}^i(h_{t+k}) - y_{t+k}^i(h_{t+k})]
    \end{aligned}
\]

Then we can rewrite the expected future wealth (the last term in the constraint) as:

\[
\sum_{s_{t+1}}a_{t+1}^i(s_{t+1},h_t)Q_t(s_{t+1}|h_t) = \sum_{k\ge 1}\sum_{h_{t+k}|h_t} q_{t+k}^t(h_{t+k})[c_{t+k}^i(h_{t+k}) - y_{t+k}^i(h_{t+k})]
\]

At $t=0$, the budget constaint becomes

$$ c_0^i(s_0) + \sum_{k\ge 1}\sum_{h_k}q_k^0(h_k)[c_k^i(h_k)-y_k^i(h_k)] = y_0^i(s_0) + a_0^i(s_0) $$

The budget constraint is exactly the same as in an Arrow-Debreu setting if we set $a_0^i(s_0)=0$.

In all consecutive future period, $t>0$, if we replace 

$$ \sum_{s_{t+1}} a_{t+1}^i(s_{t+1},h_t)Q_t(s_{t+1}|h_t) = \Omega_t^i(h_t) - [c_t^i(h_t) - y_t^i(h_t)]$$

in the budget constraint, we will get exactly the same $c_t^i$ for each $t$ as in an Arrow-Debreu equilibrium.

\subsection{Recursive Competitive
Equilibrium}

{\label{763184}}\par\null

If the state $s_t$ is governed by a Markov chain, 
$$ \pi_t(h_t) = \pi(s_t|s_{t-1})\pi(s_{t-1}|s_{t-2})\cdots\pi(s_1|s_0)\pi(s_0) $$
then the allocation for each period depend only on the current state: 
$$ y_t^i(h_t) = y^i(s_t) $$
$$ c_t^i(h_t) = c^i(s_t) $$
$$ Q_t(s_{t+1}|h_t) = Q(s_{t+1}|s_t) $$

This enables us to restate the households' utility maximisation problem
recursively.~

\par\null

For each household $i$, its Bellman equation is:

$$ V^i(a,s) = \max_{c, \hat a} \left\{ U(c) + \beta \sum_{s'}V^i(\hat a(s'), s')\pi(s'|s)\right\} $$
$$ \textrm{s.t. } c+ \sum_{s'}\hat a(s')Q(s'|s) \le y^i(s) + a $$
$$ \qquad -\hat a(s') \le \Omega^i(s') $$

The first-order conditions read:

\[
    \begin{aligned}
        c &: U'(c) - \mu = 0 \\
        \hat a &: \beta\frac{\partial V^i}{\partial\hat a}\pi(s'|s) - \mu Q(s'|s) = 0
    \end{aligned}
\]

where $\mu$ the is the Lagrangian multiplier for the first constraint. The debt limit will never bind.

By applying the Envelope theorem, 
$$ \frac{\partial V^i}{\partial a} = \mu = U'(c) $$

we have 
$$ \frac{\partial V^i}{\partial\hat a} = \mu' = U'(c') $$

Therefore, the pricing kernel can be experssed as:

$$ Q(s'|s) = \beta\frac{U'(c')}{U'(c)}\pi(s'|s) $$

\newpage 

\section{New Keynesian Model}

{\label{711332}}

\subsection{Households' Problem}

{\label{519169}}

We formulate the household's problem as following:

$$\max\mathbb{E}_t\left\{ \sum_{t=0}^{\infty}\beta^tU(C_t,N_t)\right\}$$
$$\textrm{s.t. } P_tC_t + Q_{t,t+1}A_{t+1} \le W_tN_t + A_t + \Pi_t $$

where $P_t$ stands for the aggregate price for consumption goods; $A_t$ for the net asset owned by the houdehold; and $Q_{t,t+1}$ for the relative price of the assets.

The first-order condition gives:

$$ -\frac{U_n(C_t, N_t)}{U_c(C_t, N_t)} = \frac{W_t}{P_t} $$
$$ Q_{t,t+1} = \beta\mathbb{E}_t\left\{\frac{\lambda_{t+1}}{\lambda_t}\right\} = \beta\mathbb{E}_t\left\{\frac{U_c(C_{t+1},N_{t+1})}{U_c(C_t, N_t)}\cdot\frac{P_t}{P_{t+1}}\right\}$$

If we specialize the utility function to

$$ U(C_t,N_t) = \frac{U_t^{1-\sigma}}{1-\sigma} - \frac{N_t^{1+\gamma}}{1+\gamma} $$

Then $U_c(C_t,N_t) = C_t^{-\sigma}$. Let $i_t$ be the riskless one-period nominal interest rate, $i_t$ satisfies 

$$ \frac{1}{1+i_t} = Q_{t,t+1} = \beta\mathbb{E}_t\left\{\left(\frac{C_{t+1}}{C_t}\right)^{-\sigma}\frac{P_t}{P_{t+1}}\right\} $$

Log-linearize the above equation (see \href{http://www.columbia.edu/~nc2371/teaching/R9.pdf}{What is log-linearization?}):

$$ -\hat i_t = \mathbb{E}_t \left\{-\sigma(\hat C_{t+1}-\hat C_t) - (\hat P_{t+1}-\hat P_t) \right\} $$

Rearrange it,

$$ \hat C_t = \mathbb{E}_t\hat C_{t+1} - \frac{1}{\sigma}[\hat i_{t} - \mathbb{E}_t\pi_{t+1}]$$

where we define $\pi_{t+1}=\hat P_{t+1} - \hat P_t$ as the inflation rate.

If we assume good markets clear, i.e. $\hat Y_t = \hat C_t $ where $Y_t$ represents the output, then we can rewrite the above equation as

$$ \hat Y_t = \mathbb{E}_t\hat Y_{t+1} - \frac{1}{\sigma}[\hat i_{t} - \mathbb{E}_t\pi_{t+1}]$$

This is the forward-looking Keynesian IS equation. We will return to
this equation later.~

\par\null

To facilitate the analysis of firms' pricing behavior, we assume the aggregated consumption is composed of consumptions on a list of differentiable goods:

$$ C_t =\left[\int_0^1 C_t(i) ^{\frac{\epsilon-1}{\epsilon}}\right]^{\frac{\epsilon}{\epsilon-1}}$$

Note that we aggregate the consumption bundle by a CES aggregator. 

Similarly, the price is aggregated by 

$$P_t = \left[\int_0^1(P_t(i))^{1-\epsilon}\right]^{\frac{1}{1-\epsilon}}$$

Then we can show the optimal intratemporal allocation of expenditures on each good conforms

$$ C_t(i) = \left(\frac{P_t(i)}{P_t}\right)^{-\epsilon}C_t $$

\subsection{Firms' Problem}

{\label{147524}}

We assume there is a continuum of firms on~\([0,1]\) each
producing a good differentiated by product type~\(i\).

We assume a capital-free technology:
$$ Y_t(i) = A_t N_t(i) $$

The firm minimizes its cost:

$$ \min_{N_t} W_tN_t(i) + MC_t[Y_t(i) - A_tN_t(i)]$$

where $MC_t$ is the marginal cost. 

The first-order condition is:
$$ W_t = MC_t A_t $$

\subsubsection{Marginal cost}

{\label{594939}}\par\null

Let $\displaystyle mc_t := \frac{MC_t}{P_t}$ denote the \textbf{real marginal cost} of the firm. Together with the optimal choice from the household, we can derive:

$$ mc_t = \frac{MC_t}{P_t} = \frac{W_t}{P_t}\cdot\frac{1}{A_t} = -\frac{U_n(C_t,N_t)}{U_c(C_t,N_t)}\frac{1}{A_t} = \frac{N_t^{\gamma}}{C_t^{-\sigma}}\frac{1}{A_t}$$

Assume markets clear, $C_t= Y_t=A_tN_t$, we have 

$$ mc_t = \frac{(Y_t/A_t)^{\gamma}}{Y_t^{-\sigma}}\frac{1}{A_t} = Y_t^{\sigma+\gamma} A_t^{-\gamma-1}$$

Log-linearize this equation:

$$ \hat{mc}_t = (\gamma+\sigma)\hat Y_t - (1+\gamma)\hat A_t $$

This equation establishes the real marginal cost changes as output or
technology changes.

\par\null

\subsubsection{Price stickiness}

{\label{195467}}

Now we introduce price stickiness. Suppose each period, there is a
constant probability~\(\left(1-\theta\right)\) that firms get to adjust their
prices. If firm~\(i\) gets to reset its product price at
time~\(t\), the firm knows the probability it will be stuck
at that new price for the next~\(k\) periods ahead will
be~\(\theta^k\).

The firm choose $P_t^*(i)$ to maximize expected profit:

$$ \max_{P_t(i)}\mathbb{E}_t\sum_{k=0}^{\infty} \theta^k Q_{t,t+k} C_{t+k}(i)[P_t(i) - MC_{t+k}] $$
$$\textrm{s.t. } C_{t+k}(i) = \left(\frac{P_t(i)}{P_{t+k}}\right)^{-\epsilon} C_{t+k} $$

The first-order condition gives:

$$P_t^*(i) = \frac{\epsilon}{\epsilon-1}\frac{\mathbb{E}_t\sum_{k=0}^{\infty}\theta^kQ_{t,t+k}MC_{t+k}C_{t+k}(i)}{\mathbb{E}_t\sum_{k=0}^{\infty}\theta^kQ_{t,t+k}C_{t+k}(i)}$$

If we substitute the following into the equation, $$Q_{t,t+k}=\beta^k\frac{U_c(C_{t+k}}{U_c(C_t)}\frac{P_t}{P_{t+k}} = \beta^k \Gamma_{t,t+k}$$
$$C_{t+k}(i) = \left(\frac{P_t(i)}{P_{t+k}}\right)^{-\epsilon} C_{t+k}$$
$$MC_{t+k} = mc_{t+k}P_{t+k}$$

the condition can be rewritten as:

$$P_t^*(i) = \frac{\epsilon}{\epsilon-1}\frac{\mathbb{E}_t\sum_{k=0}^{\infty}(\beta\theta)^k \Gamma_{t,t+k}mc_{t+k}P_{t+k}^{\epsilon+1}C_{t+k}}{\mathbb{E}_t\sum_{k=0}^{\infty}(\beta\theta)^k \Gamma_{t,t+k}P_{t+k}^{\epsilon}C_{t+k}}$$

Log-linearize this equation around its steady state:
$$ 
\begin{aligned}
\hat P_t^* &= \mathbb{E}_t \sum_{k=0}^{\infty}(\beta\theta)^k\frac{mc_{ss}P_{ss}^{\epsilon+1}C_{ss}}{S_{1,ss}}(\hat\Gamma_{t,t+k} + \hat{mc}_{t+k}+(\epsilon+1)\hat P_{t+k}+\hat C_{t+k}) \\
&\quad - \mathbb{E}_t\sum_{k=0}^{\infty}(\beta\theta)^k\frac{P_{ss}^{\epsilon}C_{ss}}{S_{2,ss}}(\hat\Gamma_{t,t+k} + \epsilon\hat P_{t+k}+\hat C_{t+k})
\end{aligned}
$$

where
\[
    \begin{aligned}
    S_{1,ss} &= \sum_{k=0}^{\infty} (\beta\theta)^k mc_{ss}P_{ss}^{\epsilon+1}C_{ss} = mc_{ss}P_{ss}^{\epsilon+1}C_{ss} \sum_{k=0}^{\infty} (\beta\theta)^k = mc_{ss}P_{ss}^{\epsilon+1}C_{ss} \frac{1}{1-\beta\theta}\\
    S_{2,ss} &= \sum_{k=0}^{\infty} (\beta\theta)^k P_{ss}^{\epsilon}C_{ss} = P_{ss}^{\epsilon}C_{ss} \sum_{k=0}^{\infty} (\beta\theta)^k = P_{ss}^{\epsilon}C_{ss} \frac{1}{1-\beta\theta} 
    \end{aligned}
\]

The log-linearized equation can be simplified to:

$$ \hat P_t^* = (1-\beta\theta)\mathbb{E}_t\sum_{k=0}^{\infty} (\beta\theta)^k (\hat{mc}_{t+k} + \hat P_{t+k}) $$

We can also write it in a recursive form:

$$ \hat P_t^* = (1-\beta\theta)[\hat{mc}_t + \hat P_t] + \beta\theta \mathbb{E}_t \hat P_{t+1}^* $$

\par\null

If we have enough many firms, by Law of Large Numbers, the fraction of
firms not allowed to change their prices in one-period is
approximately~\(\theta\). Therefore, the aggregate price index
can be expressed as

\par\null

$$ P_t = (\theta P_{t-1}^{1-\epsilon} + (1-\theta)(P_t^*)^{1-\epsilon})^{\frac{1}{1-\epsilon}}$$

The log-linearized form is:
$$ \hat P_t = \theta\hat P_{t-1} + (1-\theta)\hat P_t^* $$

Therefore the inflation is:
\[
\begin{aligned}
\pi_t &= \hat P_t - \hat P_{t-1} = (1-\theta)(\hat P_t^* - \hat P_{t-1}) \\
&= (1-\theta)\left[ (1-\beta\theta)(\hat{mc}_t + \hat P_t) + \beta\theta\mathbb{E}_t \hat P_{t+1}^* - \hat P_{t-1} \right] \\
&= (1-\theta)\left[ (1-\beta\theta)\hat{mc}_t + \beta\theta\mathbb{E}_t(\hat P_{t+1}^* - \hat P_t) + \pi_t \right]
\end{aligned}
\]

Hence, 
$$ \pi_t = \beta\mathbb{E}_t\pi_{t+1} + \frac{(1-\theta)(1-\beta\theta)}{\theta}\hat{mc}_t $$

This is called the New-Keynesian Phillips curve. We can read from it
that the more sticky prices are (higher~\(\theta\)), the smaller
the elasticity of inflation to marginal cost is.~~

\par\null

\subsection{Equilibrium}

{\label{481134}}

\subsubsection{Natural level of output, output
gaps}

{\label{616482}}

The natural level of output is defined as the output level when prices
are completely flexible, which implies~\(\hat{mc}_t = 0\).

Recall $$\hat{mc}_t = (\gamma+\sigma)\hat Y_t - (1+\gamma)\hat A_t$$
If $\hat{mc}_t = 0$, we get the natural level of output: 
$$ \hat Y_t^n = \left(\frac{1+\gamma}{\gamma+\sigma}\right)\hat A_t$$
i.e. the output change only depends on technological shocks.

Define the output gap, $x_t = \hat Y_t - \hat Y_t^n$. 

$$ x_t = \hat Y_t - \hat Y_t^n = \frac{\hat{mc}_t}{\gamma+\sigma} + \frac{1+\gamma}{\gamma+\sigma}\hat A_t - \frac{1+\gamma}{\gamma+\sigma}\hat A_t = \frac{\hat{mc}_t}{\gamma+\sigma} $$

Therefore, $\hat{mc}_t = (\gamma+\sigma)x_t$.

\subsubsection{New-Keynesian
equilibrium}

{\label{362724}}

We can rewrite out Phillips curve and IS curve in terms of the output
gap.~ ~

Recall the New-Keynesian Phillipse curve:

$$ \pi_t = \beta\mathbb{E}_t\pi_{t+1} + \frac{(1-\theta)(1-\beta\theta)}{\theta}\hat{mc}_t $$

Replace $\hat{mc}_t$ with $x_t$, we have

\begin{equation}
\label{eq:ps}
\pi_t = \beta\mathbb{E}_t\pi_{t+1} + \kappa x_t
\end{equation}

where $\kappa = \frac{(1-\theta)(1-\beta\theta)(\gamma+\sigma)}{\theta}$.

Recall the Keynesian IS curve:

$$ \hat Y_t = \mathbb{E}_t\hat Y_{t+1} - \frac{1}{\sigma}[\hat i_{t} - \mathbb{E}_t\pi_{t+1}]$$

Replace $\hat Y_t$ with $x_t$:
$$ x_t + \hat Y_t^n = \mathbb{E}_t(x_{t+1}+\hat Y_{t+1}^n) - \frac{1}{\sigma}[\hat i_{t} - \mathbb{E}_t\pi_{t+1}]$$

Rearrange it, we get:

\begin{equation}
\label{eq:is}
x_t = \mathbb{E}_t x_{t+1} - \frac{1}{\sigma}(\hat i_t - \mathbb{E}_t \pi_{t+1} - r_t^n)
\end{equation}

where $r_t^n = \sigma\mathbb{E}_t(\hat Y_{t+1}^n - \hat Y_t^n) = \sigma\frac{1+\gamma}{\gamma+\sigma}\mathbb{E}_t(\hat A_{t+1} - \hat A_t)$ is a representation of expected technology changes. 

The equilibrium in this New-Keynesian economy is thus defined as follows: given the stochastic process $\{\hat A_t\}$ and a monetary policy plan $\{\hat i_t\}$, a \textbf{rational expectation equilibrium} (REE) is the set of stochastic processes $\{x_t, \pi_t\}$ satisfying \eqref{eq:ps} and \eqref{eq:is}.

\subsection{Simple monetary policy}

{\label{773166}}

Consider the monetary authority following a simple decision rule
specified by the reaction function:

$$ \hat i_t =\phi_\pi \pi_t $$

Substitute the monetary policy in \eqref{eq:is}, then the REE can be characterized by

\[
\begin{bmatrix}
\pi_t \\
x_t
\end{bmatrix}
=
\underbrace{
\frac{1}{\sigma+\kappa\phi_\pi}
\begin{bmatrix}
\alpha\beta+\kappa & \sigma\kappa \\
1-\beta\phi_\pi & \sigma 
\end{bmatrix}
}_{\mathbf F}
\begin{bmatrix}
\mathbb{E}_t \pi_{t+1} \\
\mathbb{E}_t x_{t+1}
\end{bmatrix}
+ \frac{1}{\sigma+\kappa\phi_\pi}
\begin{bmatrix}
\kappa \\ 1
\end{bmatrix}
r_t^n
\]

It can be shown if $\phi_\pi \ge 1$, $\mathbf F$ is a stable matrix (all its eigenvalues are inside the unit circle), and there exists a unique stable REE $\{x_t, \pi_t\}$ satisfying \eqref{eq:ps} and \eqref{eq:is}.

\newpage

\section{Search and Matching}

{\label{749203}}

This section presents a simple model that a single unemployed worker is
faced with a contemporaneous decision of accepting a stochastic wage
offer and be employed immediately, or accepting some unemployment
benefit payout and thus remaining unemployed for one more period.~

\subsection{Wage Offer Distribution}

{\label{189039}}

Suppose wage offer is a random variable drawn from a distribution with
cumulative probability:

$$ \mathbb{P}(w \le W) = F(W) $$

We assume~\(F\) satisfying the following properties:

\begin{itemize}
\tightlist
\item
  \(F\) is nondecreasing and continuous;
\item
  There exists a finite~\(B\) such that~\(F\left(B\right)=1\)
\end{itemize}

Then the expected value of~\(w\) is:

$$ \mathbb{E}(w) = \int_0^B w dF(w) = \left[wF(w)\right]_0^B - \int_0^B F(w)dw = B - \int_0^B F(w)dw $$

\subsection{One-Sided Job Search}

{\label{158547}}

Assume a worker facing a job search decision. Given wage
offer~\(w\), the decision process is a binary action:

\begin{itemize}
\tightlist
\item
  accept~\(w\), and receive~\(w\) per period
  forever;
\item
  reject~\(w\), receive unemployment
  benefit~\(c\), and wait for next period
  offer~\(w'\).
\end{itemize}

\par\null

Let $V(w)$ be the woker's value function:

\[
V(w) = \begin{cases}
w + \beta w^2 +\cdots = \frac{w}{1-\beta}, \qquad\textrm{if accept}\\
c + \beta \int_0^B V(w') dF(w') \qquad\textrm{if reject}
\end{cases}
\]

The worker's optimal decision is to choose the action that yields the
highest payoff. Note that~\(\frac{w}{1-\beta}\) is an increasing function
in~\(w\), and~\(c + \beta \int_0^B V(w') dF(w')\) is a constant determined
by the distribution function and is irrespective of~\(w\).
So there must exists a threshold~\(\overline{w}\)~such that the worker
optimally accept the offer if~\(w \ge \overline{w}\), and reject
if~\(w < \overline{w}\).

Therefore, we can write the worker's Bellman equation as

\[
V(w) = \begin{cases}
w + \beta w^2 +\cdots = \frac{w}{1-\beta}, \quad\qquad\qquad\textrm{if }w \ge \overline{w}\\
c + \beta \int_0^B V(w') dF(w') = \frac{\overline{w}}{1-\beta} \qquad\textrm{if }w < \overline{w}
\end{cases}
\]

\(\overline{w}\) is called the \textbf{reservation wage}.

\par\null

\subsubsection{Reservation Wage}

{\label{160406}}

We argue that given distribution~\(F\), the reservation
wage~\(\overline{w}\) is uniquely determined. To prove this argument,
we arrange the second the equation in the Bellman equation:

~

\[
\begin{aligned}
 \frac{\overline{w}}{1-\beta} &= c + \beta \int_0^B V(w') dF(w') \\
 &= c + \beta \int_0^{\overline{w}} \frac{\overline{w}}{1-\beta} dF(w') + \beta \int_{\overline{w}}^{B} \frac{w'}{1-\beta} dF(w') \\
 &= c + \frac{\beta\overline{w}}{1-\beta} \int_0^{\overline{w}}dF(w') + \beta \int_{\overline{w}}^{B} \frac{w'}{1-\beta} dF(w')
\end{aligned}
\]

Therefore, 
\[  
\overline{w}\int_0^{\overline{w}}dF(w') = c + \frac{\beta}{1-\beta}\int_{\overline{w}}^{B} w' dF(w')
\]

Add $\int_{\overline{w}}^{B} \overline{w} dF(w')$ on both sides, we have:

\begin{equation}
\label{eq:rw}
\underbrace{\overline{w} - c}_{\textrm{cost of being unemployed}} = \underbrace{\frac{\beta}{1-\beta}\int_{\overline{w}}^{B} (w'-\overline{w}) dF(w')}_{\textrm{expected benefit of searching one more period}} 
\end{equation}

Let $g(w) = w - c$, and $h(w) = \frac{\beta}{1-\beta}\int_{w}^{B} (w'-w) dF(w')$.

\[
\begin{aligned}
h'(w) &= \frac{\beta}{1-\beta}[F(w)-1] < 0 \\
h''(w) &= \frac{\beta}{1-\beta}F'(w) > 0
\end{aligned}
\]

So \(h\left(w\right)\) is a decreasing convex function of
\(w\), while \(g\left(w\right)\) is an increasing function
of \(w\). Therefore there is a unique \(\overline{w}\)
that equals \(h\left(w\right)\) and \(g\left(w\right)\), which proves the
unique existence of the reservation wage.

If the unemployment compensation~\(c\)
increases,~\(g\left(w\right)\) is shifted downwards, it
follows~\(\overline{w}\) increases --- the worker's threshold
increases and demands higher wage offers.~

\par\null

\subsubsection{Mean Preserving Spread}

{\label{613325}}

What happened if we change the wage distribution~\(F\)?
Suppose we change the distribution so as to increase the probability of
getting higher wages as well as lower wages. Let's call the new
distribution~\(F'\).~\(F'\) satisfies the
following properties:

$$\int_0^B[F'(w) - F(w)]dw = 0$$
$$ F'(w) - F(w) = 
          \begin{cases}
             \le 0 \textrm{ for } w \ge \hat w\\
             \ge 0 \textrm{ for } w \le \hat w
          \end{cases}
$$

Such a redistribution represents a \textbf{mean-preserving increase in
risk} since it redistribute the mass of probability toward the tails of
the distribution while keeping the mean constant.

To see the effect such redistribution, we rearrange Equation \eqref{eq:rw}:
\[
\begin{aligned}
\overline{w} - c &= \frac{\beta}{1-\beta}\int_{\overline{w}}^{B} (w'-\overline{w}) dF(w') \\
&= \frac{\beta}{1-\beta}\int_{0}^{B} (w'-\overline{w}) dF(w') - \frac{\beta}{1-\beta}\int_{0}^{\overline{w}} (w'-\overline{w}) dF(w') \\
&= \frac{\beta}{1-\beta}[\mathbb{E}(w) - \overline{w}] - \frac{\beta}{1-\beta}\int_{0}^{\overline{w}} (w'-\overline{w}) dF(w') \\
&= \frac{\beta}{1-\beta}[\mathbb{E}(w) - \overline{w}] + \frac{\beta}{1-\beta}\int_{0}^{\overline{w}} F(w') dF(w')
\end{aligned}
\]

Therefore, 
$$ \overline{w} - c = \beta[\mathbb{E}(w) - c] + \beta\int_{0}^{\overline{w}} F(w') dF(w') $$



Let $g(w) = w - c$, and $h(w) = \beta[\mathbb{E}(w) - c] + \beta\int_{0}^{w} F(w') dF(w') $. 
\[
\begin{aligned}
h'(w) &= F(w) > 0 \\
h''(w) &= F'(w) > 0
\end{aligned}
\]
Therefore, $h(w)$ is an increasing convex function of $w$. The intersection of $g(w)$ and $h(w)$ uniquely determines $\overline{w}$.

If we replace $F$ with $F'$, note that the mean-preserving condition implies
$$ \int_0^y F'(w) dw \ge \int_0^y F(w) dw \qquad\textrm{ for } 0 \le y \le B$$

So~\(h\left(w\right)\) is shifted upward. The result is the
threshold~\(\overline{w}\) increases. The intuition is, greater
probability of high wage offers increases the value of searching,
therefore increases the threshold of accepting an offer.~

\newpage

\section{References}

{\label{891579}}

\begin{enumerate}
\tightlist
\item
  Ljungqvist, Lars, and Thomas J. Sargent.~\emph{Recursive macroeconomic
  theory}. MIT press, 2012.
\item
  Miao, Jianjun.~\emph{Economic dynamics in discrete time}. MIT press,
  2014
\item
  Kam,
  Timothy.~\href{https://phantomachine.github.io/econ8022/index.html\#}{Dynamic
  Macroeconomics}.
\end{enumerate}

\par\null\par\null

\selectlanguage{english}
\clearpage
\end{document}

