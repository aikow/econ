\documentclass[11pt,a4paper]{amsart}

\usepackage{amssymb, amsmath, amsthm, latexsym}
%\usepackage[alphabetic]{amsrefs}
%\usepackage{calrsfs}
%\usepackage{graphicx}
\usepackage{enumerate}
\usepackage{color}

\newcommand{\norm}[1]{\left\lVert#1\right\rVert}

\def\R{{\mathbb R}}
\def\Q{{\mathbb Q}}
\def\Z{{\mathbb Z}}
\def\C{{\mathbb C}}
\def\S{{\mathbb S}}
\def\N{{\mathbb N}}
\def\H{{\mathbb H}}
\def\RP{{\mathbb {RP}}}
\def\p{\partial}
\def\O{\Omega}
\def\pO{{\partial\Omega}}
\def\bO{\overline\Omega}
\def\a{\alpha}
\def\b{\beta}
\def\g{\gamma}
\def\G{\Gamma}
\def\d{\delta}
\def\D{\Delta}
\def\e{\epsilon}
\def\r{\rho}
\def\t{\tau}
\def\l{\lambda}
\def\L{\Lambda}
\def\k{\kappa}
\def\s{\sigma}
\def\th{\theta}
\def\o{\omega}
\def\z{\zeta}
\def\n{\nabla}
\def\T{{\mathcal T}}
\def\X{{\mathcal X}}
\def\area{\mathrm{area}}
\def\dist{\mathrm{\rm dist}}
\def\diag{\mathrm{diag}}
\def\spt{\mathrm{spt\,}}
\def\diam{\mathrm{diam\,}}
\def\dim{\mathrm{dim\,}}
\def\graph{\mathrm{graph\,}}
\def\interior{\mathrm{int\,}}
\def\exterior{\mathrm{ext\,}}
\def\diag{\mathrm{diag\,}}
\def\Image{\mathrm{Image}\,}
\def\osc{\mathop{\text{\rm osc}}}


\def\ra{\rightarrow}
\def\rai{\rightarrow\infty}

\overfullrule0pt \hoffset=0cm
\parskip=4pt
\parindent=0pt
\baselineskip=20pt

\begin{document}

\thispagestyle{empty}

\begin{center}
\huge
\vspace*{1.0in} MATH2320/3116/6110
\\\vspace{0.5in} \textsc{Assignment 2}
\normalsize
\\\vspace{0.5in} \textsc{By}
\\\vspace{0.1in} \textsc{Zeming Wang}
\\\vspace{0.1in} \textsc{u6114134}
\normalsize
\\\vspace{0.5in} \textsc{Lecture: John Urbas}
\\\vspace{0.1in} \textsc{Tutor: Sophie Chen}
\\\vspace{0.1in} \textsc{Tutorial: Wednesday 5-6 pm}
\normalsize
\\\vspace{0.5in} \textsc{Due: 29 March 2018}
\end{center}

\newpage
\setcounter{page}{1}

%All questions are of equal value.

%\bigskip


{\bf 1.} [4+4+4+4=16 points]

Let $(X,d)$ be a metric space.

(a) Show that  $\tilde d(x,y)=\min\{1,d(x,y) \}$ defines a metric on $X$.

\begin{proof}
  To show $\tilde{d}$ is a metric, we need to show:
  \begin{enumerate}[(i)]
    \item $\tilde{d}(x,y) \geq 0$, $\tilde{d}(x,y)=0$ iff $x=y$;
    \item $\tilde{d}(x,y) = \tilde{d}(y,x)$;
    \item $\tilde{d}(x,y) \leq \tilde{d}(x,z) + \tilde{d}(z,y)$.
  \end{enumerate}
  It is clear that (i) and (ii) hold because $d(x,y)$ is a metric.
  It only remains to show the triangle inequality holds, i.e.

  $$ \tilde{d}(x,y) \leq \tilde{d}(x,z) + \tilde{d}(z,y) $$

  If $d(x,y) \leq 1$, then $\tilde{d}(x,y) = d(x,y)$.
  If both $d(x,z) \leq 1$ and $d(z,y) \leq 1$,  we have
  $\tilde{d}(x,z) = d(x,z)$ and $\tilde{d}(z,y) = d(z,y)$.
  The triangle inequality holds for $\tilde{d}$, because
  it holds for $d$. If either $d(x,z) > 1$ or $d(z,y) > 1$, we must have
  $\tilde{d}(x,z) + \tilde{d}(z,y) > 1$. Because $\tilde d(x,z) \leq 1$,
  the triangle inequality also holds. \\

  If $d(x,y) > 1$, then $\tilde{d}(x,y) = 1$. On the right hand side,
  if both $d(x,z) \leq 1$ and $d(z,y) \leq 1$,
  $\tilde d(x,z)+\tilde d(z,y) = d(x,z)+d(z,y) \geq d(x,y) > 1 = \tilde d(x,y)$.
  If either $d(x,z) > 1$ or $d(z,y) > 1$, we have
  $\tilde{d}(x,z) + \tilde{d}(z,y) > 1 = \tilde{d}(x,y)$.
  Therefore, the triangle inequality also holds.
\end{proof}

\medskip

(b) Is $\hat d=d^2$ necessarily a metric on $X$? Give a proof or a counterexample.

\begin{proof}
No. To show a counterexample, consider the metric space $X=\R$ and $d(x,y)=|x-y|$.
Let $x > z > y$, using the $\hat d=d^2$ metric, we have
$$ |x-y|^2 = |x-z + z-y|^2 = |x-z|^2 + |z-y|^2 + 2|x-z||z-y| > |x-z|^2 + |z-y|^2 $$
So $\hat d=d^2$ is not a metric.
\end{proof}

\medskip

(c) Let $X$ be a nonempty set and suppose that $d:X\times X\ra \R$
satisfies
$$ d(x,x)=0 \quad\mbox{for all} \quad x\in X, $$
$$ d(x,y)=d(y,x)\in [1,2] \quad\mbox{for all}\quad x,y\in X, \ x\neq y. $$
Prove that $d$ is a metric on $X$.

\begin{proof}
  The nonnegativity and symmetry are trivial. The triangle inequality
  $$d(x,y) \leq d(x,z) + d(z,y), \qquad\forall x,y,z \in X $$
  must hold because if $d(x,y) > d(x,z) + d(z,y)$, with $d(x,z) \in [1,2]$
  and $d(z,y) \in [1,2]$, we would have $d(x,y) > 2$, which contradicts the
  definition of $d(x,y) \in [1,2]$. Therefore, $d$ is a metric on $X$.
\end{proof}

\medskip

(d) Let $X$ be a nonempty set and suppose that $d:X\times X\ra \R$ satisfies all the axioms of
a metric except that the triangle inequality is reversed:
$$ d(x,y) \ge d(x,z)+d(z,y) \quad\mbox{for all}\quad x,y,z\in X. $$
Prove that $X$ contains exactly one element.

\begin{proof}
  Follow the reversed triangle inequality, we have
  $$ d(x,y) \ge d(x,z)+d(z,y) $$
  $$ d(x,z) \ge d(x,y)+d(y,z) $$
  $$ d(y,z) \ge d(y,x)+d(x,z) $$
  Sum up for both sides, and use axiom of symmetry,
  $$ d(x,y) + d(x,z) + d(y,z) \ge 2(d(x,y) + d(x,z) + d(y,z) )$$
  Therefore,
  $$ d(x,y) + d(x,z) + d(y,z) \le 0 $$
  By the axiom of nonnegativity, this implies
  $$ d(x,y) = d(x,z) = d(y,z) = 0 $$
  which further implies $x = y = z$.
  The above result holds for all $x,y,z\in X$, which proves
  $X$ contains only one element.
\end{proof}

\bigskip


{\bf 2.} [5+10=15 points]

(a) Let $(X,d)$ be a metric space. Prove that the complement of any finite set $F\subset X$
is open.

\begin{proof}
  If $F^C$ is empty, i.e. $F = X$, then $F^C$ is open.
  If $F^C \neq \emptyset$, $\forall x \in F^C$, let
  $$ d_m = \min_{y \in F} \{ d(x,y) \} $$
  Then $B(x, d_m) \subset F^C$, because all points in $F$ have a distance from $x$
  greater than or equal to $d_m$. Therefore, $F^C$ is open.
\end{proof}

\medskip

(b) Let $X$ be a  set containing infinitely many elements, and let $d$ be a metric on $X$.
Prove that $X$ contains an open set $U$ such that $U$ and its complement $U^c=X\backslash U$
are both infinite.

\begin{proof}
  Note that if $x \in X$ is a limit point, then any open ball around $x$ contains infinitely many points.
  Because $\forall r_1>0$, by the definition of a limit point, $\exists x_1 \in B_{r_1}(x)$.
  Let $r_2 = \frac{1}{2}d(x,x_1)$, then $\exists x_2 \in B_{r_2}(x)$.
  Then we can define $r_3 = \frac{1}{2}d(x,x_2)$, and $\exists x_3 \in B_{r_3}(x)$...
  This process can go on forever. Thus we can find infinitely many points in a neighbourhood of $x$.\\

  If we can find two limit points $x,y \in X$, we can define $U = B(x, d(x,y)/2)$.
  $U$ is certainly an open set with infinitely many points.
  $U^C$ is also infinite because $B(y, d(x,y)/2) \subset U^C$. \\

  If $X$ contains at most one limit point, then it must contain infinitely many isolated points.
  Define $X' = \{ x\in X: x \textrm{ is an isolated point}\}$.
  Note that $X'$ and every subset of $X'$ is open because
  $\forall x \in X'$, $\exists r>0$ such that $B_r(x)=\{x\} \subset X'$.
  So it only remains to show $X'$ can be separated into two infinite disjoint subsets.\\

  If $X'$ is countable, i.e. $X' = \{x_i\}_{i\in\N}$. This can be done by separating $X'$ into
  $U = \{ X_{2i}\}_{i\in\N}$ and $X'\setminus U = \{ X_{2i+1}\}_{i\in\N}$.
  Then $U$ is open and $U$ and $U^C$ both infinite.
  If $X'$ is uncountable, by the well-ordering theorem, we can order the elements of $X'$
  into $x_1, x_2, x_3, \dots$, then we can define $U$ in the same way.
  Thus, the proposition is proved.
\end{proof}

\bigskip


{\bf 3.} [4+4+6=14 points]

(a) Prove that if $B_r(x)$ and $B_s(y)$ are two nonempty open balls in $\R^n$ and $B_r(x)=B_s(y)$,
then $x=y$ and $r=s$. In other words, in $\R^n$ every ball has a unique centre and a unique radius.

\begin{proof}
  It is equivalent to prove if either $x \neq y$ or $r \neq s$, then $B_r(x) \neq B_r(y)$.
  Suppose $x \neq y$ and $r = s$, then $d(x,y)>0$. Let $0 < \epsilon < d(x,y)$. Define
  $$ z = x + \frac{x-y}{\norm{x-y}}(r-\epsilon) $$
  Then $z \in B_r(x)$, because $d(z,x) = \norm{\frac{x-y}{\norm{x-y}}(r-\epsilon)} = r - \epsilon < r$.
  But $d(z,y)= \norm{ x-y + \frac{x-y}{\norm{x-y}}(r-\epsilon) } =\norm{x-y} + (r-\epsilon)=r + (d(x,y)-\epsilon) >r$,
  so $z \notin B_r(y)$. Therefore, $B_r(x) \neq B_r(y)$. \\

  Suppose $x=y$ and suppose without lost of generality $r<s$. Let $0<\epsilon<s-r$, and
  $$ z = x + \frac{x}{\norm{x}}(r+\epsilon) $$
  Then $d(z,x) = r+\epsilon > r$ and $d(z,y)=r+\epsilon < s$. So $z \in B_r(x)$ but $z \notin B_r(y)$,
  which implies $B_r(x) \neq B_r(y)$.\\

  Hence, we proved every ball in $\R^n$ has unique centre and radius.
\end{proof}

\medskip

(b) Is the result in (a) true if we replace $\R^n$ by an arbitrary metric space?
Give a proof or a counterexample.

\begin{proof}[Solution]
Not true. Consider the discrete metric space $(X,d)$ where
$$
d(x,y) =
\begin{cases}
  1 \qquad\textrm{if } x \neq y\\
  0 \qquad\textrm{if } x=y
\end{cases}
$$
Let $r>1, s>1$, then $B_r(x) = B_s(y) = X$ for all $x,y \in X$.
\end{proof}

\medskip

(c) Let $C[0,1]$ denote the collection of continuous real valued functions on $[0,1]$
equipped with the metric
$$ d(f,g)= \sup_{x\in[0,1]} |f(x)-g(x)|. $$
Let $C$ be the set of constant functions in $C[0,1]$.
Find the interior, exterior and boundary of $C$ in $C[0,1]$.
Justify your answers carefully.

\begin{proof}[Solution]
The interior of $C$ is $\emptyset$. Because $\forall f \in C$, and $\forall \epsilon > 0$,
we can always find a non-constant function $g$ such that $d(f,g) < \epsilon$.
For example, suppose $f(x) = a$, let $0 < r < \epsilon$, and
$g(x) = r\sin x + a$. Then $|g(x) - f(x)| = |r\sin x| \leq r < \epsilon$. \\

The boundary of $C$ is $C$ itself. The above proof shows there always exists a
non-constant function in $B_{\epsilon}(f)$, $\forall f \in C$, $\forall \epsilon > 0$.
We can also show there always some constant function $h \in B_{\epsilon}(f)$.
For example, if $f(x)=a$, let $0 < r < \epsilon$ and $h(x)=a-r$. Then
$|f(x)-h(x)|=|a-(a-r)|=r < \epsilon$. Therefore $\partial C = C$.\\

We know $C[0,1] = \interior{C} \cup \exterior{C} \cup \parital C = \exterior{C} \cup C$.
Therefore, $\exterior{C} = C[0,1]\setminus C = C^{C}$.

\end{proof}

\end{document}
