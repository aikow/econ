\documentclass{article}
\usepackage{amssymb}
\usepackage{amsthm}
\usepackage{amsmath}
\usepackage{graphicx}
\usepackage{enumerate}

\usepackage[a4paper, total={5in, 8in}]{geometry}

\newcommand{\N}{\mathbb{N}}
\newcommand{\R}{\mathbb{R}}
\newcommand{\powset}[1]{\mathcal{P}(#1)}

\title{Assignment 1}
% \author{
%     \textsc{Zeming Wang}
%     \thanks{u6114134@anu.edu.au}
% }
\date{}

\setlength\parindent{0pt}

\begin{document}

%\maketitle

\section*{Question 1}
\begin{enumerate}[(a)]
  \item  State how many elements each of the following sets contains:\\
  $\emptyset$, $\{0, 1\}$, $\{0, 0, \{1\}\}$, $\{0, 1, \{0, 1\}\}$,
  $\{\emptyset, \{0, 1\}\}$, $\{\emptyset, \{0\}, \{\{0, 1\}\}\}$
  \item Are any two of the sets in (a) identical? If so, which two?
  \item Is any of the sets in (a) a subset of any of the remaining ones?
        If so, state all such pairs.
\end{enumerate}

\emph{Solution}:
\begin{enumerate}[(a)]
  \item \begin{align*}
       \emptyset &: 0 \\
       \{0, 1\}  &: 2 \\
       \{0, 0, \{1\}\} &: 2 \\
       \{0, 1, \{0, 1\}\} &: 3 \\
       \{\emptyset, \{0, 1\}\} &: 2 \\
       \{\emptyset, \{0\}, \{\{0, 1\}\}\} &: 3\\
   \end{align*}

   \item No sets in (a) are identical.

   \item $\emptyset$ is a subset of every set.
   $\{0, 1\} \subset \{0, 1, \{0, 1\}\}$.

\end{enumerate}


\section*{Question 2}
Let $X$,$Y$ be nonempty sets and $f : X \to Y$ be a function.
\begin{enumerate}[(a)]
  \item Show that $f$ is one to one if and only if there is a function
        $g: Y \to X$ such that $g \circ f$ is the identity function on $X$.
  \item Show that $f$ is onto if and only if there is a function $g: Y \to X$
        such that $f \circ g$ is the identity function on $Y$.
\end{enumerate}

\begin{proof}
$ $\\
\begin{enumerate}[(a)]
  \item ($\Rightarrow$) Suppose $f$ is one-to-one. Let $x_0 \in X$. Define
        \[ g(y) = \begin{cases}
              x,\quad \textrm{ if } \exists x \textrm{ s.t. } f(x)=y \\
              x_0,\quad \textrm{otherwise}
           \end{cases}
        \]
        Then $g(f(x)) = x$, meaning $g \circ f$ is the identity function on $X$.

        ($\Leftarrow$) Suppose there exists $g$ such that $g(f(x)) = x$.
        If $f(x_1) = f(x_2)$, we have $g(f(x_1)) = g(f(x_2))$, which implies
        $x_1 = x_2$. Therefore, $f$ is one-to-one.

  \item ($\Rightarrow$) Suppose $f$ is onto. For every $y \in Y$, $\exists x$
        such that $f(x) = y$. Define $g(y) = x_0$ where $x_0 \in g^{-1}(y)$.
        Then $f(g(y))=y$, meaning $f \circ g$ is the identity function on $Y$.

        ($\Leftarrow$) Suppose there exists $g$ such that $f(g(y)) = y$.
        Then for any $y \in Y$, there exists $x = g(y)$ such that $f(x)=y$,
        which proves $f$ is onto.
\end{enumerate}
\end{proof}


\section*{Question 3}
Prove that it is possible to write $\N$ as a union $\N = \cup_{j=1}^{\infty} A_j$
where $A_i \cap A_j = \emptyset$ if $i\neq j$ and each $A_j$ is an infinite set.\\

\begin{proof}
It is sufficient to prove the proposition by constructing $\{ A_j \}$. \\

\emph{Construction I}: Let $A_j$ be the set of all natural numbers whose all
decimal digits sum up to $j$. For example,
\begin{align*}
  A_1 &:= \{x:\textrm{sum of all all digits of } x = 1\} =\{1,10,100,\dots\}\\
  A_2 &:= \{x:\textrm{sum of all all digits of } x = 2\}
          = \{2,20,200,\dots\, 11, 110, \dots \}\\
  \vdots
\end{align*}
This construction of $\{ A_j \}$ satisfy all the three requirements:
\begin{enumerate}[(i)]
  \item Each $A_j$ is an infinite set;
  \item $A_i \cap A_j = \emptyset$ if $i\neq j$;
  \item $\N = \cup_{j=1}^{\infty} A_j$.
\end{enumerate}
To justify (iii), just notice
(a) for any $x \in \N$, the digits of $x$ must sum up
to some natural number, thus there exists some $n$ such that $x \in A_n$,
which means $ \N \subset \cup_{j=1}^{\infty} A_j$;
(b) $\forall j, A_j \subset \N$, therefore $\cup_{j=1}^{\infty} A_j \subset \N$.
\\

\emph{Construction II}: Every natural number can be uniquely factorized to the product
of prime numbers, $x = 2^i \cdot 3^j \cdot 5^k \dots$.
Define $\{ A_j \}$ as follows:
\begin{align*}
  A_1 &:= \{ 2^i: i \in \N \} \\
  A_2 &:= \{ 2^i \cdot 3^j: i,j \in \N^+ \} \\
  A_3 &:= \{ 2^i \cdot 3^j \cdot 5^k: i,j,k \in \N^+ \} \\
  \vdots
\end{align*}
Obviously, (i) each $A_j$ is an infinite set. And because the factorization is unique
for every natural number, it is true that (ii) $A_i \cap A_j = \emptyset$ if $i\neq j$.
(iii) holds because, firstly $ \cup_{j=1}^{\infty} A_j \subset \N$; secondly,
for any $x \in \N$, $x$ can be factorized and thus $x \in A_j$ for some $j$,
so $\N \subset \cup_{j=1}^{\infty} A_j$.



\end{proof}


\section*{Question 4}
Let $X = \{ A \subset \N : A \textrm{ and } \N \setminus A \textrm{ are infinite} \}$.
Prove that $X$ is uncountable.\\


\begin{proof}
Obviously, $X \subset \powset{\N}$.  We will first prove $\powset{\N}$ is
uncountable, and $\powset{\N}\setminus X$ is countable, and conclude
$X$ is uncountable.\\

Notice that any subset $A$ of $\N$ can be represented by a string of binaries:
$(b_1, b_2, b_3, \dots)$ where $b_k = 1$ if $k \in A$ and $b_k = 0$
if $k \notin A$.\\

\emph{$\powset{\N}$ is uncountable}. To prove this,  we can use a similar
technique as Cantor's diagonalization argument.
Suppose $\powset{\N} = \{A_1, A_2, \dots\}$
is countable and thus can be listed. We now list all the elements of
$\powset{\N}$ along with their binary representation as follows:
\begin{align*}
  A_1 &: (b_{11}, b_{12}, b_{13}, \dots) \\
  A_2 &: (b_{21}, b_{22}, b_{23}, \dots) \\
  A_3 &: (b_{31}, b_{32}, b_{33}, \dots) \\
  \vdots
\end{align*}
Now we can construct a new binary representation $(c_1, c_2, c_3, \dots)$ where
$c_k = 1 - b_{kk}$ that differs from every one in the list.
Yet $(c_1, c_2, c_3, \dots)$ is a representation of some subset of $\N$,
which yields a contradiction. Hence, $\powset{\N}$ is uncountable.\\

\emph{$\powset{\N}\setminus X$ is countable}.
Notice that $\powset{\N}\setminus X = \{A \subset \N: A \textrm{ is finite} \}
\cup \{A \subset \N: \N\setminus A \textrm{ is finite} \}$.
Since the union of two finite set is finite, it is sufficient to prove
the set of all finite subset of $\N$ is countable.
Let $Y = \{A \subset \N: A \textrm{ is finite}\}$.
We can apply the same technique to list all elements of $Y$ with thier
binary representations.
\begin{align*}
  A_1 &: (b_{11}, b_{12}, b_{13}, \dots) \\
  A_2 &: (b_{21}, b_{22}, b_{23}, \dots) \\
  A_3 &: (b_{31}, b_{32}, b_{33}, \dots) \\
  \vdots
\end{align*}
Because every $A \in Y$ is finite, so its binary representation only has
finite number of $1$s. Therefore, we can define $f: Y \to \N$ as follows:
$$ f(A_i) := \sum_{j=0}^{\infty} b_{ij} 2^j $$
The inverse of $f$ is just turning an interger into its binary representation
and to an element of $Y$. Therefore, there exists a bijection between $Y$
and $\N$, and set $Y$ is countable. \\

Now, back to the original question, since $\N \setminus A$ is countable,
if $X$ is countable, $\powset{\N} = X \cup (\N \setminus A)$ must be countable,
which contradicts with the fact that $\powset{\N}$ is uncountable.
Therefore, $X$ must be uncountable.
\end{proof}

\end{document}
