\documentclass[11pt,a4paper]{amsart}

\usepackage{amssymb, amsmath, amsthm, latexsym}
%\usepackage[alphabetic]{amsrefs}
%\usepackage{calrsfs}
%\usepackage{graphicx}
\usepackage{bm}

\def\R{{\mathbb R}}
\def\Q{{\mathbb Q}}
\def\Z{{\mathbb Z}}
\def\C{{\mathbb C}}
\def\S{{\mathbb S}}
\def\N{{\mathbb N}}
\def\H{{\mathbb H}}
\def\RP{{\mathbb {RP}}}
\def\p{\partial}
\def\O{\Omega}
\def\pO{{\partial\Omega}}
\def\bO{\overline\Omega}
\def\a{\alpha}
\def\b{\beta}
\def\g{\gamma}
\def\G{\Gamma}
\def\d{\delta}
\def\D{\Delta}
\def\e{\epsilon}
\def\r{\rho}
\def\t{\tau}
\def\l{\lambda}
\def\L{\Lambda}
\def\k{\kappa}
\def\s{\sigma}
\def\th{\theta}
\def\o{\omega}
\def\z{\zeta}
\def\n{\nabla}
\def\T{{\mathcal T}}
\def\X{{\mathcal X}}
\def\area{\mathrm{area}}
\def\dist{\mathrm{\rm dist}}
\def\diag{\mathrm{diag}}
\def\spt{\mathrm{spt\,}}
\def\diam{\mathrm{diam\,}}
\def\dim{\mathrm{dim\,}}
\def\graph{\mathrm{graph\,}}
\def\interior{\mathrm{interior\,}}
\def\diag{\mathrm{diag\,}}
\def\Image{\mathrm{Image}\,}
\def\osc{\mathop{\text{\rm osc}}}


\def\ra{\rightarrow}
\def\rai{\rightarrow\infty}

%\setlength{\parindent}{0pt}

\overfullrule0pt \hoffset=0cm
\parskip=4pt
\parindent=0pt
\baselineskip=20pt

\begin{document}

\thispagestyle{empty}
\begin{center}
\huge
\vspace*{1.0in} MATH2320/3116/6110
\\\vspace{0.5in} \textsc{Assignment 4}
\normalsize
\\\vspace{0.5in} \textsc{By}
\\\vspace{0.1in} \textsc{Zeming Wang}
\\\vspace{0.1in} \textsc{u6114134}
\normalsize
\\\vspace{0.5in} \textsc{Lecture: John Urbas}
\\\vspace{0.1in} \textsc{Tutor: Sophie Chen}
\\\vspace{0.1in} \textsc{Tutorial: Wednesday 5-6 pm}
\normalsize
\\\vspace{0.5in} \textsc{Due: 7 May 2018}
\end{center}

\newpage
\setcounter{page}{1}

% \thispagestyle{empty}
%
% \begin{center}{\bf MATH2320/3116/6110 Analysis 1 2018}\end{center}
%
%
%
% \medskip
%
% \begin{center}{\bf Assignment 4 }\end{center}
%
% \medskip
%
% \begin{center}{\it Due 5pm Monday 7 May 2018}
%  \end{center}
%
% \smallskip
%
%
% {\sl Marking scheme: 45 points + 5 points for presentation. Total 50 points. }
%
% \medskip


%All questions are of equal value.

%\bigskip




{\bf 1.} [23 points]

Let $l^\infty$ denote the vector space of bounded sequences of real numbers, with addition and scalar multiplication defined componentwise.
Define a norm $\|\cdot\|$ on $l^\infty$ by
$$\|{\bm x}\|=\sup_{i\in\N}|x_i| <\infty
\quad\mbox{where ${\bm x}=(x_1,x_2,x_3,\dots)$}. $$

(a) [5 points] Prove that $l^\infty$ is complete with respect to the norm $\|\cdot\|$.

\begin{proof}
  To prove $l^\infty$ is complete, we need to show every Cauchy sequence in $l^\infty$ converges.
  Let $(\bm x_n) \in l^\infty$ be Cauchy. Then by definition,
  $\forall\e>0$, $\exists N\in\N$ such that $\forall m,n>N$, we have
  $$\| \bm x_m - \bm x_n \| = \sup_{i}|x_{m,i}-x_{n,i}|<\e $$
  Then for any $i$-th terms in $\bm x_m$ and $\bm x_n$, we have
  $$ |x_{m,i} - x_{n,i}| \le \sup_{i}|x_{m,i}-x_{n,i}|<\e $$
  This implies the sequence $(x_{n,i})_{n=1}^{\infty} \subset \R$ is Cauchy for any given $i$.
  Since $\R$ is complete, $(x_{n,i})_{n=1}^{\infty}$ converge.
  Let $x_i = \lim_{n\to\infty} x_{n,i}$.

  Then we claim: $\bm x = (x_1, x_2, \dots)$ is the limit for $\bm x_n$.

  For $m,n>N$, by triangle inequality, we have
  $$ \|\bm x_m - \bm x \| \le \|\bm x_m - \bm x_n\| + \|\bm x_n - \bm x \| $$

  Fix $m>N$ and let $n\to\infty$, then
  $$ \|\bm x_n - \bm x \| = \sup_{i} |x_{n,i} - x_i| \to 0 $$
  since $x_i = \lim_{n\to\infty} x_{n,i}, \forall i\in\N$.

  By $(\bm x_n)$ being Cauchy, we have $\|\bm x_m - \bm x_n\| < \e $. Therefore,
  $$ \|\bm x_m - \bm x \| \le \|\bm x_m - \bm x_n\| + \|\bm x_n - \bm x \| \le\e $$

  This is true for any $m>N$. Hence, we conclude $\bm x_m\to \bm x \in l^\infty$
  as $m\to\infty$ and therefore $l^\infty$ is complete.
\end{proof}
\medskip

(b) [9 points] Consider the following subspaces of $l^\infty$.
\begin{itemize}
	\item[(i)]  $c=\mbox{the space of convergent sequences}$;  \medskip

	\item[(ii)] $c_0=\mbox{the space of sequences converging to 0}$;  \medskip

	\item[(iii)] $ c_{00}=\mbox{the space of sequences which are eventually 0}$.  \smallskip

\end{itemize}

Are $c$, $c_0$, $c_{00}$ closed subspaces of $l^\infty$? Prove your answer in each case.

\begin{proof}[Proof (i)]$ $
First note $c$ is a subspace, because by the properties of convergence, we have
\begin{enumerate}
  \item If $\bm x, \bm y \in c$, then $\bm x + \bm y \in c$;
  \item If $\bm x\in c$, $\a\in\R$, then $\a\cdot\bm x \in c$;
  \item Finally, $\bm 0 = (0,0,0,\cdots) \in c$.
\end{enumerate}

$c$ is also closed. Let $(\bm x_n)\subset c$ and $\bm x_n \to \bm x$.
It is sufficient to show $\bm x \in c$, i.e. $\bm x$ is also a convergent sequence.

$\bm x_n$ being a convergent sequence implies $\bm x_n$ is Cauchy, therefore
$\forall\e>0$, $\exists K\in\N$ such that $\forall i,j>K$, we have
$$ |x_{n,i} - x_{n,j}| < \frac{\e}{3} $$

Since $\bm x_n \to \bm x$, $x_{n,i}\to x_i$ for each $i$.
Therefore $\exists N\in\N$ such that $\forall n>N$,
$$ |x_{n,i} - x_i| < \frac{\e}{3}, \quad |x_{n,j} - x_j| < \frac{\e}{3} $$

Then by triangle inequality,
$$ |x_i - x_j| \le |x_{n,i} - x_i| + |x_{n,i} - x_{n,j}| + |x_{n,j} - x_j| < \e $$
which shows $\bm x$ is Cauchy. By the completeness of $\R$, $\bm x$ is also convergent.
\end{proof}

\begin{proof}[Proof (ii)]
It is easy to see $c_0$ is also a linear subspace. $c_0$ is also closed.

Let $(\bm x_n) \subset c_0$ and $\bm x_n \to \bm x$. We will show $\bm x\in c_0$.

$\bm x_n \in c_0$ implies $\forall\e>0$, $\exists N\in\N$ such that
for $k>K$, $|x_{n,k} - 0| < \frac{\e}{2}$.

$\bm x_n \to \bm x$ implies $\exists N\in\N$ such that for $n>N$,
$|x_{n,k} - x_k| < \frac{\e}{2}$.

Then by triangle inequality, we have
$$ |x_k - 0| \le |x_{n,k} - x_k| + |x_{n,k} - 0| < \e $$
which shows $\bm x$ is also a sequence converging to $0$.
\end{proof}

\begin{proof}[Proof (iii)]
$c_{00}$ also forms a subspace, by noting that, if $\bm x \in c_{00}$,
then $\exists N_1\in\N$ such that $x_i=0$ whenever $i>N_1$; and
$\bm y \in c_{00}$ implies $\exists N_2\in\N$ such that $y_i=0$ whenever $i>N_2$.
Then we have $x_i + y_i = 0$ by taking $i> \max\{N_1, N_2\}$.
Therefore, $\bm x + \bm y \in c_{00}$.
And $\a \cdot \bm x\in c_{00}$ also holds with $\forall\a\in\R$,
just by noting $\a\cdot x_i = 0$ whenever $i > N_1$.

But $c_{00}$ is not closed. As a counterexample, consider the sequence
$$ \bm x_n = \Big( \underbrace{1,\frac{1}{2}, \cdots, \frac{1}{n}}_{n},0,0,\cdots \Big)$$
$\bm x_n \in c_{00}$ and $\bm x_n$ converges to $\bm x = \Big(1,\frac{1}{2}, \frac{1}{3}, \cdots \Big)$,
because $\| \bm x_n - \bm x \| = \frac{1}{n+1} \to 0$ as $n\to\infty$.
But $\bm x\notin c_{00}$. Therefore, $c_{00}$ is not closed.
\end{proof}
\medskip

(c) [9 points] State whether the following subsets of $l^\infty$ are (sequentially) compact.
Give reasons for your answers.

\begin{itemize}
	\item[(i)] $\displaystyle{ A = \left\{ {\bm x}=(x_n)\in l^\infty: |x_j| \le 1 \mbox{ for $j=1,\dots,k$} \right\}}$
	where $k\in \N$ is fixed;

  \begin{proof}
  $A$ is not compact. Consider the following sequence in $A$:
  \begin{align*}
    \bm x_1 &= ( \underbrace{0, 0,\cdots, 0}_{k},1,1,1,\cdots ) \\
    \bm x_2 &= ( \underbrace{0, 0,\cdots, 0}_{k},2,2,2,\cdots ) \\
            &\vdots \\
    \bm x_n &= ( \underbrace{0, 0,\cdots, 0}_{k},n,n,n,\cdots )  \\
            &\vdots
  \end{align*}
  It is clear $(\bm x_n) \subset A$. But $(\bm x_n)$ do not contain a convergent subsequence.
  Therefore, $A$ is not compact.
  \end{proof}

	\item[(ii)] $\displaystyle{ B = \left\{ {\bm x}=(x_n)\in l^\infty:  \|{\bm x}\|\le 1, \ x_n=0 \mbox{ for all } n>k \right\}}$
	where $k\in \N$ is fixed;

  \begin{proof}
  $B$ is compact. Let $(\bm x_n)$ be a sequence in $B$:
  \begin{align*}
    \bm x_1 &= ( \underbrace{x_{11}, x_{12},\cdots, x_{1k}}_{k},0,0,0,\cdots ) \\
    \bm x_2 &= ( \underbrace{x_{21}, x_{22},\cdots, x_{2k}}_{k},0,0,0,\cdots ) \\
            &\vdots \\
    \bm x_n &= ( \underbrace{x_{n1}, x_{n2},\cdots, x_{nk}}_{k},0,0,0,\cdots )  \\
            &\vdots
  \end{align*}
  Then for every $n$, let $\bm x_n^k = (x_{n1}, x_{n2}, \dots, x_{nk}) \in [-1,1]^k \subset \R^k$.
  By Bolzano Weierstrass Theorem, every bounded sequence in $\R^k$ has a convergent subsequence.
  Therefore, $(\bm x_n^k)_{n=1}^{\infty}$ has a convergent subsequence, by relabelling,
  we can also denote it as $(\bm x_n^k)$. By appending infinitely many $0$s at the end of every $\bm x_n^k$,
  we get a convergent subsequence $(\bm x_n)$ in $B$.
  \end{proof}

	\item[(iii)] $\displaystyle{C = \left\{ {\bm x}\in l^\infty:  \|{\bm x}\|\le 1 \right\}}$.

  \begin{proof}
    $C$ is not compact. Consider the sequnce $(\bm x_{n})\subset C$ where $x_{n,i} = \sin ni$.
     It is clear that $|x_{n,i}| \le 1$ and $\|\bm x\|\le 1$. But it does not contain a convergent subsequence.
     To see this, extract any subsequence from  $(\bm x_{n})$, by relabelling, call it also  $(\bm x_{n})$.
     Let $\e = 1$. For any $N\in\N$, let $m,n>N$, we have
     $$ \| \bm x_m - \bm x_n \| = \sup_{i} |x_{m,i} - x_{n,i}| = \sup_{i} |\sin mi - \sin ni| $$
     Since $\sup |\sin mi - \sin ni| = 2$, by taking $m,n$ sufficiently large, we can always
     have $\| \bm x_m - \bm x_n \| > \e$.
     Therefore, $(\bm x_n)$ is not Cauchy and therefore do not converge.
  \end{proof}
\end{itemize}



\bigskip


{\bf 2.} [10 points]

(a) [4 points] Let $(X,d)$ and $(Y,\r)$ be metric spaces, $X$ compact, and suppose  $f:X\rightarrow Y$ and
$f_n:X \rightarrow Y$ are continuous functions such that $\r(f_n(x),f(x))$ \underline{decreases}
to $0$ as $n\ra \infty$ for each $x\in X$. Prove that $f_n$ converges uniformly to $f$.

% Hint: Reduce to Dini's theorem (Theorem 12.1.3 in the notes), see also Remark (i) following the
% theorem.

\begin{proof}
  Let $g_n: X\to\R$ be defined as $g_n(x) = \rho(f_n(x),f(x))$.
  Then $(g_n)$ is a decreasing sequnce and $g_n \to 0$ pointwise as $n\to\infty$.

  To apply Dini's Theorem, we also require $g_n$ to be continuous.
  But this follows immediately from the continuity of $f_n$ and $f$:
  Let $x_n \to x \in X$. $f_n$ and $f$ being continuous implies that
  $f_n(x_n) \to f_n(x)$ and $f(x_n) \to f(x)$.
  Therefore $\rho(f_n(x_n),f(x_n)) \to \rho(f_n(x),f(x))$,
  i.e. $g_n(x_n) \to g_n(x)$.

  Given $X$ being compact, by Dini's Theorem, $g_n \to 0$ uniformly. i.e.\\
  $\rho(f_n,f) \to 0$ uniformly. Then by definition, $f_n \to f$ uniformly.
\end{proof}
\medskip

(b) [6 points] For each of the following sequences of functions, determine whether it converges
(either pointwise or uniformly), and find the limit function if it exists.
Justify your answer in each case.

\medskip

\begin{itemize}

\item[(i)] $\displaystyle{g_n(x)= \frac{1}{1+nx^2}}$,  \quad $x\in\R$.

\medskip

\item[(ii)] $\displaystyle{h_n(x)= \frac{1}{n}\cos(e^{nx})}$,  \quad $x\in\R$.

\end{itemize}

\begin{proof}
(i) It is easy to see $g_n(x)$ converges pointwise to
\[
g(x) = \begin{cases}
      1 \qquad\textrm{ if } x = 0 \\
      0 \qquad\textrm{ if } x\neq 0
    \end{cases}
\]
Because for $x=0$, $g_n(x)=1$; for $x\neq 0$, $g_n(x)\to 0$ as $n\to\infty$.

But the convergence is not uniformly.

Let $\e \in (0,1)$. $\forall N\in\N$,
fix $n>N$. By making $x$ sufficiently small, in particular,
$$ |x| < \sqrt{\frac{1}{n}\left(\frac{1}{\e}-1\right)} $$
and $x\neq 0$, we have $g(x)=0$, and
$$ |g_n(x) - g(x)| = \frac{1}{1+nx^2} > \e $$
Therefore, $g_n$ does not converge to $g$ uniformly.\\

(ii) $h_n(x)$ converge to $h(x)=0$ uniformly.

For any $\e>0$, let $N = \frac{1}{\e}$, then for $n>N$, we have
$$ |h_n(x) - 0| = \frac{1}{n} |\cos(e^{nx})| \le \frac{1}{n} < \e $$
Therefore $h_n(x)\to 0$ as $n\to\infty$ uniformly.
\end{proof}

\medskip



\bigskip


{\bf 3.} [12 points]

Let $(X,d)$ and $(Y,\r)$ be metric spaces and $f:X\rightarrow Y$.

\medskip

(a)  [4 points] Prove that $f$ is uniformly continuous on $X$ if and only if for any sequences
$(x_n)$ and $(z_n)$ in $X$,
$$ d(x_n,z_n) \rightarrow 0 \quad\Longrightarrow\quad \r(f(x_n),f(z_n)) \rightarrow 0. $$

\begin{proof} $ $

  $(\Rightarrow)$ Suppose $f$ is uniformly continuous. Let $(x_n)$ and $(z_n)$ be two sequences
  with $d(x_n, z_n) \to 0$. Then $\forall \e>0$, $\exists \d>0$ such that
  $d(x_n, z_n) < \d$ $\Rightarrow$ $\rho(f(x_n), f(z_n)) < \e$.
  This shows $d(x_n, z_n) \to 0 \Rightarrow \rho(f(x_n), f(z_n)) \to 0$.\\

  $(\Leftarrow)$ Suppose $d(x_n, z_n) \to 0 \Rightarrow \rho(f(x_n), f(z_n)) \to 0$,
  but suppose $f$ is not uniformly continuous.
  Then $\exist\e>0$, $\forall\d>0$, $\exists x,z \in X$ such that
  $d(x,z) < \d$ but $\rho(f(x), f(z)) \ge \e$.
  Fix such $\e$, let $\d = \frac{1}{n}$. Then for any $n\in\N$,
  $\exists x_n,z_n \in X$ such that $d(x_n, z_n) < \frac{1}{n}$
  but $\rho(f(x_n), f(z_n)) \ge \e$.
  Then we constructed two sequences $(x_n)$ and $(z_n)$ with $d(x_n,z_n)\to 0$,
  but $\rho(f(x_n), f(z_n)) \not\to 0$ for all $n$ --- a contradiction.
\end{proof}
\medskip

(b)  [3 points] Prove that if $f:X\rightarrow Y$ is uniformly continuous, then $f$ maps
Cauchy sequences to Cauchy sequences, i.e.,
$$ (x_n) \mbox{ is Cauchy in $X$} \quad\Longrightarrow\quad
(f(x_n))  \mbox{ is Cauchy in $Y$}. \eqno(*) $$

\begin{proof}
  $f$ being uniformly continuous implies $\forall\e>0$, $\exists\d>0$ such that
  $$ d(x_1, x_2) < \d \Rightarrow \rho(f(x_1), f(x_2)) < \e $$
  Let $(x_n)$ be Cauchy, then for this $\d$, $\exists N\in\N$ such that
  $$\forall m,n>N \Rightarrow d(x_m, x_n) < \d $$
  But this implies $\rho(f(x_m), f(x_n))<\e$ for $m,n>N$.

  Therefore, $(f(x_n))$ is Cauchy in $Y$.
\end{proof}
\medskip


(c) [3 points] Is $(*)$ true if $f$ is continuous but not uniformly continuous?
Prove or provide a counterexample.

\begin{proof}
$(*)$ does not necessarily hold if $f$ is continuous but not uniformly continuous.
For example $f(x) = \frac{1}{x}$ is continuous on $(0,\infty)$ but not uniformly continuous.
Let $x_n = \frac{1}{n}$. It is easy to see $(x_n)$ is Cauchy, but $(f(x_n))$ diverge.
\end{proof}
\medskip


(d) [2 points] Is $(*)$ true if $f$ is continuous and $(X,d)$ is complete?
Give a reason for your answer.

\begin{proof}
  Yes. For an illustration, let $(x_n)$ be Cauchy in $X$. Since $(X,d)$ is complete,
  $(x_n)$ also converge, say, $x_n \to x$. Then by continuity of $f$,
  we must also have $f(x_n) \to f(x)$.
  So $(f(x_n))$ is a convergent sequence in $Y$ and is therefore Cauchy.
\end{proof}
\bigskip


\end{document}
