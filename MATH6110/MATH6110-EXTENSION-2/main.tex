\documentclass[12pt,a4paper]{amsart}

\newcommand{\suc}[1]{#1^+}
\newcommand{\bbN}{\mathbb{N}}
\newcommand{\bbQ}{\mathbb{Q}}
\newcommand{\bbZ}{\mathbb{Z}}
\newcommand{\turnin}{\faPencilSquareO}


\usepackage[charter]{mathdesign}
\usepackage{XCharter}
\usepackage[margin=1in]{geometry}
\usepackage{paralist}
\usepackage{fontawesome}
\usepackage{amssymb,amsmath,amsthm}

\title{Analysis I Advanced Extension 2018: Assignment 2}
\author{By Zeming Wang}

\begin{document}
\maketitle

\subsection*{Problem 1: Incompleteness of $\bbQ$}
In this problem we will show that $\mathbb{Q}$ is not a complete field: namely, we will find a Cauchy sequence in $\mathbb{Q}$ that does not converge in $\mathbb{Q}$.

Define a sequence $(a_n)$ recursively, as follows.
Set $a_0 = 2$.
For each $n \geq 0$, set
\[
  a_{n+1} = \frac{a_n}{2} + \frac{1}{a_n}.
\]
\begin{compactenum}[(a)]
  \setlength{\itemsep}{0.5em}
\item \turnin\ Show that $a_n^2 > 2$ for each $n\in \bbN$.

\begin{proof}
Prove by induction. $a_n^2 > 2$ holds for $n=0$, since $a_0=2$.
Suppose $a_n^2 > 2$ holds for $n=k$. For $n=k+1$, we have
$$ a_{k+1}^2 = \left(\frac{a_k}{2} + \frac{1}{a_k}\right)^2
           = \frac{a_k^2}{4} + \frac{1}{a_k^2} + 1
           \ge 2\sqrt{\frac{a_k^2}{4}\cdot \frac{1}{a_k^2}} + 1
           = 1+1 = 2
$$
The equality holds only if $\dfrac{a_k^2}{4} = \dfrac{1}{a_k^2}$
i.e. $a_k^2 = 2$. But $a_k^2 > 2$  by induction hypothesis,
therefore $a_{k+1}^2 > 2$.
Hence, $a_n^2 > 2$ for each $n\in \bbN$.
\end{proof}


\item \turnin\ Show that for each $n \geq 0$, we have
  \[
    0 < \left(\frac{a_n}{2} - \frac{1}{a_n}\right) < \frac{1}{2^n}.
  \]

\begin{proof}
By part (a), we have $a_n > 0$ and $a_n^2 > 2$ for all $n\in \bbN$.
$$ a_n^2 > 2 \Leftrightarrow \frac{a_n}{2} > \frac{1}{a_n} $$
Therefore, $\left(\dfrac{a_n}{2} - \dfrac{1}{a_n}\right)>0$ is proved.\\

\noindent
Prove the other part by induction. For $n=0$, the inequality holds because
$$ 0 < \left(\frac{a_0}{2} - \frac{1}{a_0}\right) = \frac{1}{2}<\frac{1}{2^0} $$
Suppose the proposition holds for $n=k$, i.e.
\begin{align*}
0 < \left(\dfrac{a_k}{2} - \dfrac{1}{a_k}\right) < \dfrac{1}{2^k}
\Leftrightarrow
0 < a_k - \left(\dfrac{a_k}{2} + \dfrac{1}{a_k}\right) < \dfrac{1}{2^k}
\Leftrightarrow
0 < (a_k - a_{k+1}) < \dfrac{1}{2^k}
\end{align*}
Therefore, $a_{k+1} < a_k$ and $\dfrac{1}{a_{k+1}} > \dfrac{1}{a_k}$.
Now consider $n=k+1$, we have
\begin{align*}
\left(\frac{a_{k+1}}{2} - \frac{1}{a_{k+1}}\right)
  < \left(\frac{a_{k+1}}{2} - \frac{1}{a_{k}}\right)
  = \frac{1}{2}\left(\frac{a_{k}}{2} + \frac{1}{a_{k}}\right) - \frac{1}{a_k}
  = \frac{1}{2}\left(\frac{a_{k}}{2} - \frac{1}{a_{k}}\right)
  < \frac{1}{2^{k+1}}
\end{align*}
Therefore, the proposition holds for $n=k+1$ and thus for all $n \in\bbN$.
\end{proof}

\item \turnin\ Conclude that $a_n$ is a decreasing sequence.
  That is, for each $n \in \bbN$, we have $a_{n+1} < a_n$.

\begin{proof}
By part (b), we have $\left(\dfrac{a_n}{2} - \dfrac{1}{a_n}\right)>0$ for all $n$. Therefore,
$$ a_{n+1} - a_n = \frac{a_n}{2} + \frac{1}{a_n} - a_n = \frac{1}{a_n} - \frac{a_n}{2} < 0 $$
which shows $a_n$ is a decreasing sequence.
\end{proof}

\item \turnin\ Show that $a_n$ is a Cauchy sequence.

\begin{proof}
By part (b), we have $0 < a_n - a_{n+1} < \dfrac{1}{2^n}$. \\
\noindent
Suppose $N \in\bbN$, $\forall n > m > N$, we have
\begin{align*}
|a_m - a_n| &\le |a_{m} - a_{m+1}| + |a_{m+1} - a_{m+2}| + \cdots + |a_{n-1} - a_{n}|  \\
&< \frac{1}{2^m} + \frac{1}{2^{m+1}} + \cdots + \frac{1}{2^{n-1}}
= \frac{1}{2^m} \left(1 + \frac{1}{2} + \cdots + \frac{1}{2^{n-m-1}} \right)
\end{align*}
The sequence $1 + \dfrac{1}{2} + \cdots + \dfrac{1}{2^{n-m-1}}$ is bounded, and
$\dfrac{1}{2^m} \to 0$ as $N\to\infty$. Therefore, $|a_m - a_n| \to 0$ as $N\to\infty$,
which shows $a_n$ is Cauchy.
\end{proof}

\item \turnin\ Show that the sequence $(a_n^2)$ converges to $2 \in \bbQ$.

\begin{proof}
By part (b), we have $\left(\dfrac{a_n}{2} - \dfrac{1}{a_n}\right) < \dfrac{1}{2^n}$.
Therefore, $\left(\dfrac{a_n}{2} - \dfrac{1}{a_n}\right)^2 < \dfrac{1}{2^{2n}}$.
$$ |a_{n+1}^2 - 2| = \left\rvert \frac{a_n^2}{4} +\frac{1}{a_n^2} - 1 \right\rvert
     = \left\rvert \left(\dfrac{a_n}{2} - \dfrac{1}{a_n}\right)^2 \right\rvert
     < \frac{1}{2^{2n}}
$$
Therefore, $|a_{n+1}^2 - 2|\to 0$ as $n\to\infty$, which is equivalent to say $a_n^2 \to 2$.
\end{proof}

\item \turnin\ Suppose that $(a_n)$ converges to a limit $q \in \bbQ$.
  Prove explicitly that we must have $q^2 = 2$.
  \emph{(Hint: you may show that for each rational $\epsilon > 0$, we have $|q^2-2| < \epsilon$.)}

\begin{proof}
\setlength{\parskip}{0.5em}
Fix $\epsilon > 0$. Let $\epsilon_1 = \dfrac{\epsilon}{2(\sqrt{2}+q)}$, $\epsilon_2 = \dfrac{\epsilon}{2}$.

\begin{compactenum}[(i)]
  \setlength{\itemsep}{0.5em}
  \item $a_n \to q$ implies $\exists N_1 > 0$, such that
        $\forall n>N_1$ $\Rightarrow$ $|a_n - q| < \epsilon_1$.
  \item By part (e), $a_n^2 \to 2$ implies $\exists N_2 > 0$ such that
        $\forall n>N_2$ $\Rightarrow$ $|a_n^2 - 2| < \epsilon_2$.
\end{compactenum}
\noindent
Therefore, by taking $n>\max\{ N_1, N_2\}$, we have
\begin{align*}
|q^2 - 2| &\le |a_n^2 - q^2| + |a_n^2 - 2| = |(a_n+q)(a_n - q)| + |a_n^2 - 2|  \\
 &< |a_n+q|\epsilon_1 + \epsilon_2
  <(\sqrt{2} + q)\cdot \frac{\epsilon}{2(\sqrt{2}+q)} + \frac{\epsilon}{2}
  = \epsilon
\end{align*}
Since $\epsilon$ can be arbitrarily small, we have $q^2=2$.
\end{proof}

\item Using the fact that there is no rational number $q$ such that $q^2 = 2$ (you don't need to prove this), conclude that $\bbQ$ is not complete.

\begin{proof}
By part (d), $(a_n)$ is Cauchy in $\bbQ$, however $a_n \to q \notin\bbQ$. Thus $\bbQ$ is not complete.
\end{proof}

\end{compactenum}

\subsection*{Problem 2: The $p$-adic norm}
For this problem, you may use the unique factorization property of integers and its consequences.
Recall that unique factorization means that every positive integer has a unique (up to order) expansion as a product of prime powers.

Fix a prime number $p$.
We recall the definition of the $p$-adic distance.
Observe that for any $x \in \bbQ$, there is a unique integer $a$ such that
\[
  x = p^a \frac{r}{s},
\]
where $r, s \in \bbZ$ such that $p \nmid r$ and $p \nmid s$.
In this case, We set $|x|_p = p^{-a}$.
Given $x, y \in \bbQ$, we define the $p$-adic distance as $d_p(x,y) = |x-y|_p$.
\begin{compactenum}[(a)]
\item \turnin\ Show that $d_p$ is a metric.
  Moreover show that it satisfies the strong triangle inequality: if $x, y\in \bbQ$, then $|x+y|_p \leq \max\{|x|_p,|y|_p\}$.

  \begin{proof}
    To show $d_p$ is a metric, we need to show:
      \begin{enumerate}
        \item $|x-y|_p \ge 0$, and $|x-y|_p = 0$ iff $x=y$;
        \item $|x-y|_p = |y-x|_p$;
        \item $|x-y|_p \le |x-z|_p + |z-y|_p$.
      \end{enumerate}
      (a) is clear since for any $x \in \bbQ$ there is a unique $|x|_p = p^{-a}$ and
      $|x|_p = p^{-a} > 0$ if $x\neq 0$, $|x|_p = 0$ if $x=0$ by definition.
      Therefore, $|x-y|_p \ge 0$ for any $x,y\in\bbQ$ and $|x-y|_p = 0$ iff $x=y$.\\
      (b) is also clear since $(x-y)$ and $(y-x)$ only differs in sign, and they
      share the same $p^a$ for some $a$. If $x=y=0$, the equality also holds
      since $|x-y|_p = |y-x|_p = 0$. \\
      (c) if we can prove the strong triangle inequality, (c) will just follow. Because if
      $$ |x+y|_p \leq \max\{|x|_p,|y|_p\} \leq |x|_p + |y|_p $$
      Then we have
      $$ |x-y|_p = |x-z+z-y|_p \leq |x-z|_p + |z-y|_p $$
      So we only need to prove that the strong triangle inequality holds.
      To show this, first note that if $x=0$ or $y=0$ or $x=y=0$, the inequality trivally holds.
      So let $x\neq 0$ and $y\neq 0$ and suppose $x=p^a \frac{r}{s}$ and $y=p^b\frac{r'}{s'}$
      where $p \nmid r,s$ and $p \nmid r',s'$.
      Without loss of generality, we may assume $a\ge b$, and thus $|x|_p \le |y|_p$. Then,
      $$ x+y = p^a \frac{r}{s} + p^b\frac{r'}{s'} = p^b\left(p^{a-b} \frac{r}{s} + \frac{r'}{s'}\right)  $$
      If $x+y = p^c \frac{r''}{x''}$ for some $c$, it must be $c\ge b$.
      Therefore, we have $|x+y|_p \le p^{-b} = \max\{|x|_p,|y|_p\}$.
  \end{proof}

\item \turnin\ Define a sequence $a_n\colon \bbN \to \bbQ$ by setting $a_0 = 1$ and $a_{n+1} = a_n + p^n$.
  Show that $a_n$ is a Cauchy sequence that converges in $\bbQ$ under the $p$-adic metric.
  Find its limit (with proof).

  \begin{proof}
    Note that since $a_{n+1} - a_n = p^n$, we have $|a_{n+1} - a_n|_p = p^{-n}$.
    Suppose $m>n>N$, by triangle inequality proved in part (a), we have
    \begin{align*}
      |a_m - a_n|_p &\le |a_m - a_{m-1}|_p + \cdots + |a_{n+1} - a_n|_p \\
                    &= p^{-(m-1)} + p^{-(m-2)} + \cdots + p^{-(n+1)} + p^{-n} \\
                    &= p^{-n} \left(1+ \frac{1}{p} + \frac{1}{p^2} + \cdots + \frac{1}{p^{m-n-1}}\right)
                     = p^{-n} \sum_{k=0}^{m-n-1} \frac{1}{p^k}
    \end{align*}
    We know the series $\sum_{k=0}^{m-n-1} \frac{1}{p^k}$ is bounded, and $p^{-n}\to 0$ as $n\to \infty$.
    Therefore, as $N\to\infty$, $|a_m - a_n|_p \to 0$, which proves $(a_n)$ is Cauchy. \\
    Note $a_n = 1+1+p+p^2+\cdots+p^{n-1}$, we would guess $a_n\to 1+\frac{1}{1-p}$
    (since the geometric series converge to $\frac{1}{1-p}$). To prove this,
    \begin{align*}
      \left\rvert a_n - \left(1 + \frac{1}{1-p}\right) \right\rvert_p &=
          \left\rvert a_{n} - a_{n-1} + \cdots + a_1-a_0 - \frac{1}{1-p}\right\rvert_p \\
          &\le |a_{n} - a_{n-1}|_p + \cdots + |a_1-a_0|_p + \frac{1}{1-p} \\
          &= p^{-(n-1)}+\codts + p^{-1} + 1 +  \frac{1}{1-p} \\
          &= \frac{1-p^{-n}}{1-p^{-1}} + \frac{1}{1-p} \\
          &\to \frac{p}{p-1} + \frac{1}{1-p} = 0 \qquad\textrm{ as } n\to\infty
    \end{align*}
    Therefore, we conclude $\lim_{n\to\infty} a_n = 1+\frac{1}{1-p}$.
  \end{proof}
\end{compactenum}
\end{document}
