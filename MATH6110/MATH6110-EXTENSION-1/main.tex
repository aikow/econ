\documentclass[12pt,a4paper]{amsart}

\newcommand{\suc}[1]{#1^+}
\newcommand{\bbN}{\mathbb{N}}
\newcommand{\turnin}{\faPencilSquareO}
\newcommand{\set}[1]{\{#1\}}

\usepackage[charter]{mathdesign}
\usepackage{XCharter}
\usepackage[margin=1in]{geometry}
\usepackage{paralist}
\usepackage{fontawesome}

\title{Analysis I Advanced Extension 2018: Assignment 1}
\author{Due on 16 March 2018\\Instructor: Asilata Bapat}
\begin{document}
\maketitle

\noindent Questions marked with a \turnin\ symbol are to be turned in for credit.
The others are just for practice and fun.
(Be careful; some of these are tricky!)
The parts are weighted equally.

\subsection*{Problem 1: Warm-up}

\begin{compactenum}[(a)]
  \setlength{\itemsep}{0.5em}
%\item Show that if $n \in \bbN$, then $\suc{n} \neq 0$.
\item[(b)] \turnin\ Show that if $n \in \bbN$, then $\suc{n} \neq n$.

\begin{proof}
By definition $\suc{n} = n \cup \{n\}$.
Suppose, for a contradiction, $\suc{n} = n$, then $n \cup \{n\} = n$.
This would mean $\{n\} \subset n$. But this cannot happen.
By the axiom of foundation, $n \cap \{n\} = \emptyset$.
If $\{n\} \subset n$, $n \cap \{n\} = \{n\} \neq \emptyset$.
Therefore, it must be $\suc{n} \neq n$.
\end{proof}

\item[(c)] \turnin\ Give an example of some $n \in \bbN$ and a proper subset $x \subsetneq n$, such that $x \notin n$.
  You should write out both $n$ and $x$ fully by listing their elements.

\begin{proof}
  Let $n = 2 = \set{ \emptyset, \set{\emptyset} }$. Let $x = \set{\set{\emptyset}}$.
  Then clearly, $x \subsetneq n$, but $x \notin n$.
\end{proof}

\item[(d)] \turnin\ Show that for every set $x$, we have $x \notin x$.

\begin{proof}
Consider the set $\set{x}$. By the axiom of foundation, there must be an element in $\set{x}$
that is disjoint with itself. Since $x$ is the only element in the set, we must have $x \cap \set{x} = \emptyset$.
Notice that if $x \in x$, $x \cap \set{x} \neq \emptyset$. So we must have $x \notin x$.
\end{proof}

\end{compactenum}

\vspace{1em}
\subsection*{Problem 2: Every $n \in \bbN$ has at most one predecessor}
The aim of this problem is to prove the theorem we stated in class: if $m, n \in \bbN$ such that $\suc{m} = \suc{n}$, then $m = n$.
\begin{compactenum}[(a)]
  \setlength{\itemsep}{0.5em}
\item \turnin\
  Let $T = \{n \in \bbN \mid \forall x \in n, x \subseteq n\}$.
  Show that $T = \bbN$.

\begin{proof}
Prove by induction.
Let $P(n)$ be the proposition: $\forall x \in n, x \subseteq n$.\\
$P(0)$ is vacuously true. Suppose $P(n)$ is true. We want to show $\forall x \in \suc{n}, x \subseteq \suc{n}$.\\
If $x \in \suc{n} = n \cup \set{n}$, either $x \in n$, or $x \in \set{n}$.
If $x \in n$, by induction hypothesis, $x \subseteq n \subseteq \suc{n}$.
If $x \in \set{n}$, then $x=n$, we also have $x \subseteq \suc{n}$.
So $x \subseteq \suc{n}$ in both cases.
Therefore we conclude $P(n)$ is true for all $n \in \bbN$, i.e. $T=\bbN$.
\end{proof}

\item \turnin\ Now let $T = \{n \in \bbN \mid \forall m\in \bbN, \text{ if } \suc{m} = \suc{n} \text{ then } m = n\}$.
  Use the previous part of the problem to show that $T = \bbN$.

\begin{proof}
Suppose $m,n \in \bbN$ and $m \neq n$. Then either $m \in n$ or $n \in m$
(since $\forall x \in \bbN$, $x \in \suc{x}$, and for any $m \neq n$, we can always establish the relation
$(\suc{(\suc{(\suc{m})})})\cdots = n$ or $(\suc{(\suc{(\suc{n})})})\cdots = m$).
If $m \in n$, then we have $\set{m} \subseteq n$. Also by (a), $m \subseteq n$. So $\suc{m} = m \cup \set{m} \subseteq n$.
Since $n \subsetneq \suc{n}$ ($n \neq \suc{n}$ proved in 1(b)), we have $\suc{m} \subsetneq \suc{n}$.
In the other case, if $n \in m$, similarly, we can prove $\suc{n} \subsetneq \suc{m}$. In either case, $\suc{m} \neq \suc{n}$.
Therefore we conclude: $\forall m,n \in \bbN$, if $m \neq n$, then $\suc{m} \neq \suc{n}$. Hence, $T = \bbN$ is proved.
\end{proof}

\end{compactenum}

\vspace{1em}
\subsection*{Problem 3: Ordinal numbers}
A set $\alpha$ is called an ordinal number if the following properties hold.
\begin{compactitem}
\item For every $x$, if $x \in \alpha$ then $x \subseteq \alpha$.
\item Every non-empty subset $T \subseteq \alpha$ has a smallest element with respect to the relation $\in$.
  That is, there is some $m \in T$ such that for every $x \in T$, either $m \in x$ or $m = x$.
\end{compactitem}
\vspace{0.5em}
\begin{compactenum}[(a)]
  \setlength{\itemsep}{0.5em}
\item \turnin\ Show that $\emptyset$ is an ordinal number.

\begin{proof}
The statement is vacuously true, since $\emptyset$ has no element or subset.
\end{proof}

\item \turnin\ Show that if $\alpha \neq \emptyset$ is an ordinal number, then $\emptyset \in \alpha$.

\begin{proof}
$\alpha$ is an ordinal number, so the two properties hold. Let the non-empty subset $T$ be $\alpha$ itself.
The 2nd property ensures there exists $m \in \alpha$ such that for every $x \in \alpha$, either $m \in x$ or $m=x$.
Take $x=\emptyset$, clearly, $\emptyset \in \alpha$. Since $m \notin \emptyset$, it must be $m=\emptyset$.
Therefore, the existence of $m \in \alpha$ guarantees $\emptyset \in \alpha$.
\end{proof}

% \item Show that if $\alpha$ is an ordinal number and $\beta \in \alpha$, then $\beta$ is an ordinal number.
% \item Show that if $\alpha\neq \beta$ are ordinal numbers and $\alpha \subseteq \beta$, then $\alpha \in \beta$.\\
%   \emph{(Remark: Compare with part (c) of Problem 1.)}
% \item Show that if $\alpha\neq \beta$ are ordinal numbers, then either $\alpha \in \beta$ or $\beta \in \alpha$.
% \item Show that every $n \in \bbN$ is an ordinal number.
%   Use this to show that $\bbN$ is an ordinal number.
% \item Can you construct an ordinal number $\beta$ such that $\bbN \in \beta$?

\end{compactenum}

\end{document}
